\nonumchap{Abstract}
\vspace{-2cm}
A quadrotor is a type of Unmanned Aerial Vehicle that has received an increasing amount of attention recently with many applications being actively investigated. 
Possible applications include search and rescue, surveillance, supply of food and medicines in emergency situations and object manipulation in construction and transportation.
An interesting subproblem of load transportation is the control of the position of a cable suspended load. The challenge is in the fact that the quadrotor-load system is highly nonlinear and under-actuated. The load cannot be controlled directly and has a natural swing at the end of each quadrotor movement. 

The goal of this thesis is to present a nonlinear geometric control approach, investigate its possibilities and limitations to track the position of a cable suspended load. The focus lies on the quadrotor-load subsystem where the cable tension is non-zero, which is analogous to modeling a rigid link between the quadrotor and load.
%The main goal is to investigate the possibilities and limitations of nonlinear geometric control for the purpose of load trajectory tracking of a quadrotor with a cable-suspended load, by evaluating the stability of the system and the tracking performance on different load trajectories. 

After introducing the basic concepts, an introduction is given on geometric mechanics. This differential geometric based approach is used to model and control the system, based on the geometric properties of the system dynamics. 
It is shown how the configuration of the quadrotor-load system can be described on smooth nonlinear geometric configuration spaces, and analyzed with the principles of differential geometry. 
This allows for modeling in an unambiguous coordinate-free dynamic fashion, while avoiding the problem of singularities that would occur on local charts. \\
Next, the geometric properties are utilized to define tracking error functions on these same spaces, making it possible to design almost-globally defined nonlinear geometric controllers. A backstepping approach is applied to generate a cascaded structure with multiple nonlinear geometric controllers, allowing control of several flight modes that are responsible for the control of 1) quadrotor attitude, 2) load attitude and 3) load position.\\ 
Finally, simulations demonstrate the stability and abilities of the nonlinear geometric controller. A Linear Quadratic Regulator is derived to compare both of the control performances. The tracking performances of both controllers are discussed for three different experiments. 
From the results of the experiments can be concluded that the nonlinear control design allows control of multiple states with the final objective of controlling the load position. The nonlinear geometric approach proves effective in load position tracking of complex trajectories and fast maneuvers,
while maintaining stability of the closed-loop system. 
%The thesis is concluded with a summary and recommendations for future work.
 

%%CHANGE meer power / eye cather, wat gebeurt er in deze thesis
%Different cases are tested to investigate the possibilities and limitations of nonlinear geometric control. 
%%ADD Results show
%Results show that....
%
%
%***************************************\\
%Intro about GC.. 
%Reasons to consider GC..
%
%***************************************\\
%Where simple linear control methods are restricted to small angle movements, nonlinear control methods allow more aggressive and faster movements.  The goal of the project is to investigate the effects on load position tracking performance when the system is modeled and controlled via a nonlinear geometric control approach.
%
%
%***************************************\\
%The quadrotor-load system is modeled in a compact and coordinate-free fashion which allows the inherent geometric properties of the system to be controlled. 
%%BOVENSTAAND IS VAAG?

%
%The main goal of this thesis is to research the effects on a cable-suspended load transportation using quadrotors, by involving complex or aggressive maneuvering through implementation of Non-Linear geometric control.
%Where linear control methods are restricted to small angle movement, non-linear control methods allow more aggressive movements. 

%
%		The main goal of this literature study is to review researches that have been done regarding the possibilities of manipulation and transportation of cable-suspended loads using quadrotors, possibly involving complex or aggressive maneuvering. Where linear control methods are restricted to small angle movement, non-linear control methods allow more aggressive movements, but are subject to complex 
%		***************************************\\
%		What is complex? If that results in super easy feedback loop, why not consider it?
%		
%		***************************************\\
%		mathematics and are more computationally intensive and less robust. 


%***************************************\\
%In the considered research papers, different modeling methods and control techniques are applied which are suitable for various specific applications. The models, derivation and underlying assumptions are studied and explained in detail. 		
%
%Super general. Can be placed in every paper.
%
%***************************************\\

%Furthermore, the studied control techniques are explained and their advantages are addressed. Several trajectory generation approaches and the related optimization techniques are studied. Their applications, with different purposes such as obstacle avoidance, time-optimal and swing-free trajectory planning are explained. The survey is concluded with a discussion about finding a suitable
%***************************************\\
%Define suitable
%
%***************************************\\
%control design to achieve the quadrotor-assisted task involving manipulation of a cable-suspended load.
%
%***************************************\\


		
\cleardoublepage