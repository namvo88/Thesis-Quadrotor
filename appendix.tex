\chapter{Appendix}

\section{Derivation of Equations of motion}
\subsection{Load Dynamics}\label{sec.app:loaddyn}
%PROOF prop.3 Sreenath2013a. Also Sreenath2013b?
%DEFINE e_3 / R / f 

Let \lsymb{$ x_{CM} $}{Position \lsymb{$ CM $}{Center of Mass} of \a{qr}-Load system} denote the position of the center of mass of the combined Quadrotor-Load system, expressed in \IF. Which can be found by
\begin{align}\label{eq:CM}
\begin{split}
m_Q(x_Q-x_{CM})+m_L(x_L-x_{CM})&=0\\
(m_Q+m_L)x_{CM}&=m_Qx_Q+m_Lx_L
\end{split}
\end{align}
Applying the laws of motion to (\ref{eq:CM}) and inserting (\ref{eq:mod.xQ2xL}) gives the 
\begin{align}\label{key}
\begin{split}
(m_Q+m_L)\ddot{x}_{CM}&=fRe_3 - (m_Q+m_L)ge_3\\
%&=fRe_3 - (m_Q+m_L)ge_3\\
%(m_Q+m_L)\ddot{x}_{CM}&=m_Q\ddot{x}_Q + m_L\ddot{x}_L\\
%\\
%m_Q\ddot{x}_Q+m_L\ddot{x}_L&=fRe_3 - (m_Q+m_L)ge_3\\
%m_Q(\ddot{x}_L-L\ddot{q})+m_L\ddot{x}_L&=fRe_3 - (m_Q+m_L)ge_3\\
(m_Q+m_L)(\ddot{x}_L+ge_3)&= fRe_3+m_QL\ddot{q}
\end{split}
\end{align}

%ADD derivation of ddq (TANG2014)


\section{LQR controller}\label{app:lqr}

\subsection{Modeling}
%From Newton's laws follows
%\begin{align}
%x_Q&=fRe_3-m_Qge_3-Tq\\
%x_L&=-m_Lge_3+Tq
%\end{align}
%
%$ x_Q $ and $ x_L $ are related by
%\begin{equation}\label{key}
%x_L = x_Q+Lq
%\end{equation}

%CHECK is dit nodig?
% From Lagrangian equations of motion, 
%
%\begin{equation}\label{eq:app.QRpos}
%\begin{aligned}
%%ADD Not done yet
%\ddot{x}&=\\
%\ddot{y}&=\\
%\ddot{z}&=
%\end{aligned}
%\end{equation}
%
%\begin{equation}\label{eq:app.QRatt}
%\begin{aligned}
%%ADD Not done yet
%\ddot{\phi}&=\\
%\ddot{\theta}&=\\
%\ddot{\psi}&=
%\end{aligned}
%\end{equation}
%
%\begin{equation}\label{eq:app.Latt}
%\begin{aligned}
%%ADD Not done yet
%\ddot{\phi}_L&=\\
%\ddot{\theta}_L&=
%\end{aligned}
%\end{equation}
%The second order system of ODEs can be transformed 
The linearized model is written into a first order ODE of the form
\begin{align}\label{eq:app.ss}
\mathbf{\dot{x} }&=A\mathbf{x}+Bu\\
y&=C\mathbf{x}+Du
\end{align}
with the following state- and input vectors
\begin{equation}\label{key}
\begin{aligned}
\textbf{x}&=\begin{bmatrix}
\textbf{q}\\
\mathbf{\dot{q}}
\end{bmatrix}\\
\mathbf{q}&=\begin{bmatrix}
x&y&z&\phi&\theta&\psi&\phi_L&\theta_L
\end{bmatrix}^T\\
\mathbf{\dot{q}}&=\begin{bmatrix}
\dot{x}&\dot{y}&\dot{z}&\dot{\phi}&\dot{\theta}&\dot{\psi}&\dot{\phi}_L&\dot{\theta}_L
\end{bmatrix}^T\\
u&=\begin{bmatrix}
f&M_\phi&M_\theta&M_\psi
\end{bmatrix}^T
\end{aligned}
\end{equation}

The model is linearized about the hovering flight mode. All translational and rotational velocities are zero during hover. The positional states and the yaw angle do not affect the dynamics, and are set equal to zero. A thrust input $ u_1=g(mQ+mL) $ is required to maintain hover, and all other control inputs are set equal to zero. 
The states and inputs in the equations of motion are substituted by an initial condition and a perturbation
\begin{equation}\label{key}
\mathbf{\dot{x}}\rightarrow\mathbf{\dot{x}}_0+\delta\mathbf{\dot{x}}, \quad \mathbf{{x}}\rightarrow\mathbf{{x}}_0+\delta\mathbf{{x}}, \quad u\rightarrow u_0+\delta u
\end{equation}
\begin{equation}\label{key}
\begin{aligned}
\mathbf{x}(0) &= \mathbf{0}\\
u(0)&=\begin{bmatrix}
g(m_Q+m_L) &0 &0& 0
\end{bmatrix}^T
\end{aligned}
\end{equation}
The linearized equations of motion are rearranged into Equation \ref{eq:app.2ode} and substituted in Equation \ref{eq:app.ss}.
\begin{equation}\label{eq:app.2ode}
\begin{bmatrix}
content...
\end{bmatrix}
\begin{bmatrix}
		\delta \ddot{x} \\\delta\ddot{y}\\\delta\ddot{z}\\\delta\ddot{\phi}\\\delta\ddot{\theta}\\\delta\ddot{\psi}\\\delta\ddot{\phi}_L\\\delta\ddot{\theta}_L 
		\end{bmatrix}+
\begin{bmatrix}
content...
\end{bmatrix}
\begin{bmatrix}
		\delta {x} \\\delta{y}\\\delta{z}\\\delta{\phi}\\\delta{\theta}\\\delta{\psi}\\\delta{\phi}_L\\\delta{\theta}_L 
\end{bmatrix}
=
\begin{bmatrix}
content...
\end{bmatrix}
\begin{bmatrix}
\delta u_1\\\delta u_2\\\delta u_3\\\delta u_4
\end{bmatrix}
\end{equation}

\begin{lstlisting}
LQRA =

Columns 1 through 8

0         0         0         0         0         0         0         0
0         0         0         0         0         0         0         0
0         0         0         0         0         0         0         0
0         0         0         0         0         0         0         0
0         0         0         0         0         0         0         0
0         0         0         0         0         0         0         0
0         0         0         0         0         0         0         0
0         0         0         0         0         0         0         0
0         0         0    9.7008         0         0    0.1092         0
0         0         0         0   -9.9217         0         0    0.1117
0         0         0         0         0         0         0         0
0         0         0         0         0         0         0         0
0         0         0         0         0         0         0         0
0         0         0         0         0         0         0         0
0         0         0    6.9291         0         0   -6.9291         0
0         0         0         0    7.0870         0         0   -7.0870

Columns 9 through 16

1.0000         0         0         0         0         0         0         0
0    1.0000         0         0         0         0         0         0
0         0    1.0000         0         0         0         0         0
0         0         0    1.0000         0         0         0         0
0         0         0         0    1.0000         0         0         0
0         0         0         0         0    1.0000         0         0
0         0         0         0         0         0    1.0000         0
0         0         0         0         0         0         0    1.0000
0         0         0         0         0         0         0         0
0         0         0         0         0         0         0         0
0         0         0         0         0         0         0         0
0         0         0         0         0         0         0         0
0         0         0         0         0         0         0         0
0         0         0         0         0         0         0         0
0         0         0         0         0         0         0         0
0         0         0         0         0         0         0         0
\end{lstlisting}
\begin{lstlisting}
LQRB =

0         0         0         0
0         0         0         0
0         0         0         0
0         0         0         0
0         0         0         0
0         0         0         0
0         0         0         0
0         0         0         0
0   -0.1591         0         0
0         0   -0.1627         0
0.2252         0         0         0
0   12.1951         0         0
0         0   11.8343         0
0         0         0    7.2622
0   10.0905         0         0
0         0   10.3203         0
\end{lstlisting}

\texttt{Matlab} command \texttt{lqr(A,B,Q,R)} generates the following gain matrix $ K $
\begin{lstlisting}
K =

Columns 1 through 8

-0.0000   -0.0000   47.6731    0.0000   -0.0000    0.0000   -0.0000    0.0000
3.1623    0.0000   -0.0000    7.5117    0.0000   -0.0000    2.6683   -0.0000
-0.0000   -3.1623   -0.0000    0.0000    7.4605    0.0000   -0.0000    2.5448
0.0000   -0.0000    0.0000   -0.0000    0.0000    1.0000    0.0000    0.0000

Columns 9 through 16

-0.0000   -0.0000   21.1202    0.0000    0.0000    0.0000   -0.0000   -0.0000
2.7501    0.0000   -0.0000    1.4827    0.0000    0.0000    0.4376   -0.0000
-0.0000   -2.7296    0.0000    0.0000    1.4631    0.0000   -0.0000    0.3949
0.0000   -0.0000    0.0000    0.0000   -0.0000    1.1293   -0.0000    0.0000
\end{lstlisting}




%	    ***************************************\\
%	    \begin{equation}
%	    ax = 
%	    \begin{bmatrix}
%	    f(ms_{\phi_q}s_{\psi_q}+ m_Ls_{\phi_q}s_{\psi_q} + mc_{\phi_q}c_{\psi_q}s_{\theta_q} + m_Lc_{\phi_q}c_{\psi_q}s_{\theta_q}- m_Lc_{\phi_q}c_{\psi_q}c_{\theta}^2s_{\theta_q} + m_Lc_{\phi}^2c_{\phi_q}c_{\psi_q}c_{\theta}^2s_{\theta_q} - m_Lc_{\theta}^2s_{\phi_q}s_{\psi_q}+ m_Lc_{\phi}^2c_{\theta}^2s_{\phi_q}s_{\psi_q}  + m_Lc_{\psi_q}c_{\theta}s_{\phi}s_{\phi_q}s_{\theta} + m_Lc_{\phi}c_{\phi_q}c_{\theta_q}c_{\theta}^2s_{\phi} - m_Lc_{\phi_q}c_{\theta}s_{\phi}s_{\psi_q}s_{\theta_q}s_{\theta})
%	    
%	    
%	    (   + Lmm_Lv_\theta^2c_{\theta}s_{\phi} + Lmm_Lv_\phi^2c_{\theta}^3s_{\phi} )/(m(m + m_L))\\
%	    \\
%	    (Lmm_Lv_\theta^2s_{\theta} - fm_Lc_{\psi_q}c_{\theta}^2s_{\phi_q} - fmc_{\psi_q}s_{\phi_q} + fmc_{\phi_q}s_{\psi_q}s_{\theta_q} + fm_Lc_{\phi_q}c_{\theta}^2s_{\psi_q}s_{\theta_q} + Lmm_Lv_\phi^2c_{\theta}^2s_{\theta} + fm_Lc_{\phi}c_{\phi_q}c_{\theta_q}c_{\theta}s_{\theta} - fm_Lc_{\theta}s_{\phi}s_{\phi_q}s_{\psi_q}s_{\theta} - fm_Lc_{\phi_q}c_{\psi_q}c_{\theta}s_{\phi}s_{\theta_q}s_{\theta})/(m(m + m_L))\\
%	    \\
%	    -(gm^2 + gmm_L - fmc_{\phi_q}c_{\theta_q} - fm_Lc_{\phi_q}c_{\theta_q} + Lmm_Lv_\theta^2c_{\phi}c_{\theta} + fm_Lc_{\phi}^2c_{\phi_q}c_{\theta_q}c_{\theta}^2 + Lmm_Lv_\phi^2c_{\phi}c_{\theta}^3 + fm_Lc_{\phi}c_{\psi_q}c_{\theta}s_{\phi_q}s_{\theta} - fm_Lc_{\phi}c_{\theta}^2s_{\phi}s_{\phi_q}s_{\psi_q} - fm_Lc_{\phi}c_{\phi_q}c_{\theta}s_{\psi_q}s_{\theta_q}s_{\theta} - fm_Lc_{\phi}c_{\phi_q}c_{\psi_q}c_{\theta}^2s_{\phi}s_{\theta_q})/(m(m + m_L))\\
%	    \\
%	    (- Lmc_{\theta}s_{\theta}v_\phi^2 + fc_{\psi_q}c_{\theta}s_{\phi_q} - fc_{\phi}c_{\phi_q}c_{\theta_q}s_{\theta} - fc_{\phi_q}c_{\theta}s_{\psi_q}s_{\theta_q} + fs_{\phi}s_{\phi_q}s_{\psi_q}s_{\theta} + fc_{\phi_q}c_{\psi_q}s_{\phi}s_{\theta_q}s_{\theta})/(Lm)\\
%	    \\
%	    -(fc_{\phi_q}c_{\theta_q}s_{\phi} + fc_{\phi}s_{\phi_q}s_{\psi_q} - 2Lmv_\phi v_{\theta}s_{\theta} + fc_{\phi}c_{\phi_q}c_{\psi_q}s_{\theta_q})/(Lmc_{\theta})
%	    \end{bmatrix}
%	    \end{equation}	    
%	    
%	    ***************************************\\	

\subsection{Controller}

\begin{equation}\label{key}
A=\begin{bmatrix}
content...
\end{bmatrix}
\end{equation}
\begin{equation}\label{key}
B=\begin{bmatrix}
content...
\end{bmatrix}
\end{equation}
\begin{equation}\label{key}
C=\begin{bmatrix}
content...
\end{bmatrix}
\end{equation}

%CHECK if needed?
\begin{equation}\label{key}
\mathbf{\dot{x}}=\textbf{f}(\mathbf{x,u})
\end{equation}

%ADD reference
%By linearizing Equations \ref{eq:app.QRpos},\ref{eq:app.QRatt},\ref{eq:app.Latt} follows
%\begin{align}\label{key}
%%CHECK this equations
%\ddot{x}&=-\frac{f}{m}\\
%\ddot{y}&=\\
%\ddot{z}&=
%\end{align}

%The mathematical model is linearized around the following operating points
%\begin{align}\label{key}
%\bar{\mathbf{x}}&=\begin{bmatrix}\bar{x}&\bar{y}&\bar{z}&\textbf{0}_{1\times13}\end{bmatrix}^T\\
%\bar{\textbf{u}}&=\begin{bmatrix}
%(m_Q+m_L)g&0&0&0
%\end{bmatrix}^T
%\end{align}

%CHECK are these asssumptions correct?
%Assuming small angles, the following holds
%\begin{align}\label{key}
%\text{for } \gamma &= \phi, \theta, \psi, \theta_L, \psi_L\\
%sin(\gamma)&\simeq \gamma\\
%cos(\gamma)&\simeq 1\\
%\dot{\gamma} &\simeq 0\\
%F &\simeq (m_Q+m_L)g
%\end{align}

%\begin{equation}\label{key}
%A=\frac{\partial \textbf{f}(x,u)}{\partial x}\mid _{	x=\bar{x},u=\bar{u}	}
%\end{equation}
%\begin{equation}\label{key}
%B=\frac{\partial \textbf{f}(x,u)}{\partial u}\mid _{	x=\bar{x},u=\bar{u}	}
%\end{equation}
%
%\begin{equation}\label{key}
%u=-K\left[\textbf{x}_{des}(t)-\textbf{x}(t)\right] 
%\end{equation}


\section{Figures}
\begin{figure}[h!]
	\centering
	\makebox[\textwidth][c]{\includegraphics[width=.45\textwidth]{./StyleStuff/dcsc.png}}
	\caption{Simulink Command Filter\label{fig:app.CF}}
\end{figure}		

\section{\texttt{MATLAB} code}
\begin{equation}\label{key}
Test equation
\end{equation}
\subsection{A \matlab Listing}

\lstset{language=matlab}
\lstinputlisting{test.m}

%    \subsection{An appendix subsection with C++ Listing}
%
%    \lstset{language=C++}
%    \lstinputlisting{test.c}    

%    \chapter{Appendix: Figures}
%
%    \section{Test section (again?)}
%
%    Ok, all is well.