\chapter{Conclusions and Future Work}\label{ch:conclusion}
\section{Summary and Conclusions}

\section{Recommendations for Future Work}\label{ch:future}
% DONT make it look like literature survey. Based on details, show what I learned. What could be interesting next. What is missing for the next steps.
***************************************\\
Model Validation; now it's estimation/copied from other work\\ 
System Identification

***************************************\\
\subsection{Modeling Constraints}
There are several techniques to handle input saturation, the most popular ones are anti-windup techniques. Back-calculation is such a method for PID to activate the integrator, is this possible for NL control?

***************************************\\
\cite{Goodarzi2013a} includes uncertainties in the translational dynamics and rotational dynamics. Out of the scope, might be interesting.\\
***************************************\\

\subsection{Hybrid Modeling}
Switching between several flight modes yields autonomous acrobatic maneuvers. Robust to switching conditions ***why?\\
\cite{Tang2014}

\subsection{Trajectory Generation}
\subsubsection{Minimum Snap Trajectory Generation}

Trajectory can be generated by solving a \a{QP} via minimum snap generation.

Problem in smaller 4-D space instead of 12-D, with help of differential flatness. Explain differential flatness and its usefulness.

Is able to include constraints in \a{QP}.

\cite{Mellinger2011}

Obstacle avoidance