\chapter{Conclusions and Future Work}\label{ch:conclusion}
\section{Summary and Conclusions}
% %MODEL
This thesis aims to give insight in a nonlinear geometric approach for the control of a quadrotor with a cable suspended load. 
In order to test the concept of the nonlinear geometric approach, a mathematical model is composed by applying techniques from differential geometry. 
Differential geometry is the mathematical discipline that is used to study these geometric problems.
This method allows the model to be defined globally on nonlinear manifolds, which is not possible on Euclidean spaces.
The configuration space of the system is described on nonlinear manifolds, instead of applying Euclidean spaces that are defined by Cartesian coordinates.
With the tools of differential geometry, differential calculus and integral calculus, a globally defined model is obtained.

%%CONTROL
Nonlinear geometric control is used for the design of a control system for this dynamic system that evolves on nonlinear manifolds.
The model allows nonlinear geometric controllers to define error functions tot calculate error values that evolve on the same nonlinear manifolds, 
through matrix operations that arise from linear algebra. 
A backstepping approach allows different \a{DOF}s of the under-actuated system to be controlled.
The controllers consists of control laws that guarantee stabilization of the closed-loop system in a cascaded structure by using the states as virtual control inputs. 

% %EXPERIMENTS
The stability and performance of the nonlinear geometric controller is tested in a set of experiments, each describing different load trajectories. 
Experiments are simulated and the nonlinear control performance is compared to the proven concept of an \a{lqr} controller.
The nonlinear tracking controller proves almost-global exponential attractiveness to the zero equilibrium of the tracking errors. 

% %CONCLUSIE
% wat lost het op?
%wat is het voordeel?
%Result/conclusie;

%Wat is goed, wat is niet goed? Waar komt dat door?
%What is its value of nonlinear control compared to linear control

The nonlinear geometric approach allows the control of multiple states with the final objective of controlling the load position.
From the results of the experiments can be concluded that the nonlinear control design has proven effective in tracking complex trajectories and fast maneuvers,
while maintaining closed-loop stability of the system. \\
Furthermore, the advantage that the controllers are almost-globally defined, makes them suitable for complex load position tracking.
The controller shows overall satisfying results for the load position tracking experiments. 
%However, tuning the controller is not an intuitive process. 

Tuning of the controller can be done by checking the system responses and the error functions.
Tuning the controllers for the \a{qr} and load attitude controlled mode can be done in a similar fashion as tuning a simple \a{pd} controller. 
Adding proportional gain results in a faster response, where adding derivative gain for a better damping. 
However, tuning the load position controlled mode is no longer intuitive due to the nonlinearities and coordinate changes by the control laws.

Where the nonlinear geometric controller employs a nonlinear model that is defined on nonlinear manifolds,
the \a{lqr} controller employs a model that is defined and linearized in a Euclidean space. 
The nonlinear geometric approach allows the control of the dynamics that are described by the nonlinear model.
The linear controller has an incomplete model due to unmodeled dynamics, and the controller does not allow direct tracking of a load trajectory.
For that reason, the \a{lqr} controller is tuned by making a trade-off must be made between sufficient \a{qr} position tracking and minimization of the load swing. 

From the comparison between the controllers it can be concluded that, 
near the equilibrium configuration, the \a{lqr} controller is able to reduce the swing relatively fast and accurate.
%The controller was able to steer the \a{qr} to large angles, but loses
However, the shortcomings of the \a{lqr} controller become evident when it is given the task to track complex trajectories. 
The controller is not able to cope with the fast changing dynamics, resulting in overshoots and slow responses, causing the load to lag behind the desired trajectory.
Keep in mind that the comparison with an \a{lqr} controller is meant as a proof of concept of \a{ngc}, and the goal is not to optimize the \a{lqr} controller. 


%The controllers are functions of the computed tracking references $ q_c, R_c $ and their derivatives. These terms are approximated by a command filter, which means that the accuracy decreases because high frequency terms are filtered.

%Too high gains will create a fast but aggressive responses to errors. 
%Too small gains however will result in slow responses, creating large errors in the load position.

%controls the \a{qr} rotations in order to influence the load attitude dynamics. 
%That makes the nonlinear geometric controller suitable for 
%The \a{ngc} controller is designed to track a load position, where 

%This illustrates the differences 
%The largest differences between the \a{lqr} and \a{ngc} controllers lie in the fact that 
%that the main difference between the controllers is caused 

%Following these trajectories continuously requires fast changing movements, far away from the equilibrium configuration. 

%The performance of the \a{lqr} controller depends on the tuning of the cost function.
%The \a{lqr} cost function allows control of the states that define the \a{qr} position, \a{qr} attitude and load attitude by calculating the control inputs $ f $ and $ M $, in such a way that the system is stable. 
%The control can be optimized for minimization of rotation, translation or load swing. 

%To adjust the effects of states and control inputs, the controller can be tuned intuitively by adjusting the weight matrices in the cost function.

%The \a{lqr} control commands small rotations in order to succeed in both position tracking of the \a{qr} and minimization of the load swing.
%The \a{ngc} control commands large \a{qr} rotations in order to create a large influence on the load attitude dynamics, see Equation \ref{eq:mod.loadatt}. 
%commands small rotations in order to succeed in both position tracking of the \a{qr} and minimization of the load swing.



\section{Recommendations for Future Work}\label{ch:future}
% dont make it look like literature survey. Based on details, show what I have learned. What could be interesting next. What is missing for the next steps.
%Lack of time: no implementation possible. 
%In order to realize this, one needs to investigate the subjects
\subsection{Investigate Implementation}
\paragraph{Digital control} 
The concept of geometric control is shown under the assumption of continuous-time control. 
However, an analysis must be done in the discrete-time domain for the implementation of a real-time application. 
The feasibility to run the controller on an on-board processor on a \a{qr} must be investigated. 
The control performance could be limited by the bandwidth of either the discretized control system or the wireless communication.
It must be investigated whether the control system is still able to deal with the fast dynamics that are required for aggressive maneuvering. 
Continuous-time Euler-Lagrange equations were found by minimizing the action integral, which is a function of the Lagrangian. 
In a similar procedure the discrete-time Euler-Lagrange can be obtained, by minimizing te summation of a discrete Lagrangian, which is demonstrated in \cite{Lee2008}.\\
Furthermore, command filters were applied to approximate the derivatives of the command inputs and the parameter choice was arbitrarily. 
An extension to the use of the filters, is the investigation of the parameter choice and its effects.
However, the implementation of the command filters was also done under the assumption of continuous-time control. 
The calculations can be achieved by implementation of simple low-pass filters, potentially saving computational power and possibly interesting for an on-board implementation. However, it must be investigated whether this approach preserves the geometric properties of a mechanical system and controller stability in the discrete-time domain.
\paragraph{Model identification, validation and robustness}
In this thesis the model parameters are either obtained from examples in literature or arbitrarily chosen. 
In practice, identification and validation of the \a{qr} model and rotor dynamics is required.\\
The control in this thesis assumes perfect state feedback. In practice the controller depends on visual feedback or data obtained from an on-board inertial measurement unit. Unlike in simulations, this data will contain noise, uncertainties and possibly drift. \\
Based on a nonlinear geometric approach for a \a{qr} without load, \cite{Goodarzi2013a} includes uncertainties in the translational dynamics and rotational dynamics to prove robustness. As a theoretical extension, this approach could be extended to a \a{qr}-load system to test the influence of model mismatches and robustness of the controller. 
\paragraph{Hybrid System Control}
This thesis is focused on the subsystem where the tension in the cable is non-zero.
The dynamics of the \a{qr}-load system will be better covered by adding the subsystem where the cable tension is zero. 
More specifically, the second subsystem simply considers a \a{qr} with a disconnected load in free fall.
Both subsystems can be modeled via the same nonlinear geometric approach.\\
A possible extension is to investigate the possibilities to apply hybrid control, 
such that the controller is able to switch between control for each of the two subsystem depending on whether the cable tension is zero or non-zero.
In \cite{Sreenath2013c,Tang2014,Tang2015} both subsystems are expressed in the form of one hybrid nonlinear geometric model and a trajectory generation method that accounts for the switching dynamics of the hybrid system is presented. 

\subsection{Trajectory Generation}
\paragraph{Minimum Snap Trajectory Generation}
The trajectories described in Section \ref{sec:exp.traj} were arbitrarily generated by hand to test the performance of the controller in different situations. 
Recall that the nonlinear geometric controllers require a twice-differentiable trajectory. Whenever more complex or optimal trajectories are required, manual generation of trajectories is no longer efficient and become too complex to solve by hand. A recommended extension to this thesis is the automatic generation of a trajectory.\\ 
Trajectory generation algorithms exist that are able to generate a smooth desired position, velocity and acceleration by solving a \acs{qp} optimization problem. 
This approach is presented by \cite{Mellinger2011} and applied in \cite{Tang2014,Tang2015}. 
The \a{qp} even allows inclusion of constraints on inputs or trajectory in the optimization.
Furthermore, it is proven that the system is \textit{differential flat}, meaning that all states and inputs can be expressed in terms of only four states and their derivatives. This property is used to transform the high-dimensional optimization problem into a four-dimensional problem.
