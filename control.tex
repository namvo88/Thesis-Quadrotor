\chapter{Control Design} \label{ch:control}

Geometric Control Theory explores the application of differential geometric techniques to systems control. The objective is to express both the dynamics and its control input on manifolds instead of on local charts.

Geometric Control is based on a coordinate-free representation of the dynamics. The equations of motion are compact, unambiguous and singularity free. 

Attitude control systems naturally evolve on non-linear configurations such as $ S^2 $ and $ SO(3) $. 

Global nonlinear dynamics of various classes of closed loop attitude control systems have been studied in recent years \cite{Chaturvedi2011a}.

***************************************\\


%ADD Reason to consider Geometric Control 
Existing Control Systems for Quadrotor UAVs:\\
Based on the linearized dynamics of a quadrotor UAV \\
Singularities in representing complex maneuvers \\
Fundamental restriction in tracking nontrivial trajectories

In contrast to hybrid control systems \cite{Gillula2010}, \textbf{complicated reachability set analysis is not required} to guarantee safe switching between different flight modes, as the region of attraction for each flight mode covers the configuration space almost globally.
Tracking control system can be developed on $ SO(3) $, therefore it avoids singularities of Euler-Angles.

Benchmark with Linear Control\\

***************************************\\



***************************************\\
Therefore, the prerequisites for its study are linear algebra, vector calculus, differential geometry and non-linear control theory.
Geometric control theory is the study of how the geometry of the state space influences controls problems. This includes local properties like curvature, and global properties like the number of `holes’ in the space (sphere vs doughnut).

Intuition:  For example, I want to write out the equations of motion of a 3D pendulum and also control its orientation directly on the Special Orthogonal Group (using rotation matrices)  instead of using local charts induced by Euler Angle Parametrizations ( theta, phi and psi - the three famous angles of dynamics!).
Pros:\\
->  Compact expressions that are also intuitive.
-> Globally defined (no singularities!). Therefore, one can build almost globally attractive controllers.\\
Cons:\\
-> The math gets very involved very quickly. (This could be subjective)
***************************************\\




Control Schemes\\
Nonlinear Geometric Control. Error functions. Tracking controllers.\\

Attitude Controller\\

Define errors associated with the attitude dynamics of the QR. The attitude and angular velocity tracking error should be carefully chosen as they evolve on the tangent bundle of  $ SO(3) $. \cite{Lee 2010c} 

How to select gains for errors\\
Constants \cite{Lee2010} or matrices; enabling unique gains for roll/pitch/yaw \cite{Mellinger2011}\\

Parameter Estimation\\
State Estimation\\

\section{Backstepping Control}

***************************************\\
Nonlinear control: Backstepping control is common.
Definition backstepping: A technique for designing stabilizing controls for a special recursive class of nonlinear dynamical systems.
%The backstepping approach determines how to stabilize the {\displaystyle \mathbf {x} } \mathbf {x}  subsystem using {\displaystyle z_{1}} z_{1}, and then proceeds with determining how to make the next state {\displaystyle z_{2}} z_{2} drive {\displaystyle z_{1}} z_{1} to the control required to stabilize {\displaystyle \mathbf {x} } \mathbf {x} . Hence, the process "steps backward" from {\displaystyle \mathbf {x} } \mathbf {x}  out of the strict-feedback form system until the ultimate control {\displaystyle u} u is designed.

In a backstepping control approach, the control law is designed by using states as virtual control signals. At each step, a 

***************************************\\

\section{Geometric Control of QR system}

Control input \cite{Lee2011}
\begin{equation}\label{eq:inputattitude}
u=-k_Re_R-k_\Omega\Omega-mg\rho\times R^Te_3
\end{equation}
Insert into Equation \ref{eq:eomrigidbody}; closed loop dynamics are given by
\begin{align}\label{eq:CLdynamics}
J\dot{\Omega} &= -\Omega\times J\Omega-k_Re_R-k_\Omega\Omega \\
\dot{R} &= R\hat{\Omega}
\end{align}

***************************************\\
where is $ e_\Omega $?

***************************************\\

Three Flight Modes \cite{Lee,Lee2010c}\\
Attitude Controlled Mode: track an attitude command $ R_d(t) $ \\
Position Controlled Mode: track a position command $ x_d(t) $ and a heading direction $ b_{1_d}(t) $\\
Velocity Controlled Mode: track a velocity command $ v_d(t) $ and a heading direction $ b_{1_d}(t) $\\

		\subsection{Attitude Tracking}
		***************************************\\
		Asymptotic tracking of the quadrotor attitude does not require specification of the thrust magnitude. As an auxiliary problem, the thrust magnitude can be chosen in many different ways to achieve an additional translational mo- tion objective. For example, it can be used to asymptotically track a quadrotor altitude command [28]. Since the translational motion of the quadrotor UAV can only be partially controlled; this flight mode is most suitable for short time periods where an attitude maneuver is to be completed. \cite{Goodarzi2015b}
		
		***************************************\\

		\subsection{Position Tracking}
		
		
		Explain how $ f $ and $ \vec{b_{1_d}} $ is obtained from $ x_d(t) $?\\
		
		The attitude and angular velocity tracking error should be carefully chosen as the evolve on the tangent bundle of the nonlinear space $ SO(3) $. \cite{Lee2010}\\
		Why? Appendix\cite{Lee2010}, and \cite{Bullo2005}?
		
		The error function on $ SO(3) $ is chosen to be \cite{Lee2010}
		\begin{equation}\label{eq:errorfunc}
		\Psi(R,R_d)=\frac{1}{2}tr\left[I-R_d^TR\right]
		\end{equation}

\section{Geometric Control of QR-Load system}
Three Flight Modes\\
QR Attitude Controlled Mode: track a QR attitude command $ R_d(t) $ and a heading direction $ b_{1_d}(t) $ \\
Load Attitude Controlled Mode: track a load attitude command $ q_d(t) $ \\
Load Position Controlled Mode: track a load position $ x_{L,d}(t) $\\

		\subsection{Quadrotor Attitude Tracking}

		\subsection{Load Attitude Tracking}
		
		\subsection{Load Position Tracking}

\section{Parameter - and State Estimation}
How to choose parameters and how to estimate states?


\section{Conclusion}

***************************************\\
What is Geometric Control?\\
Why Geometric Control?\\
Control design will be based on Nonlinear Geometric Control\\

The proposed control system is robust to switching conditions since each flight mode has almost global stability properties, and it is straightforward to design a complex maneuver of a QR. \cite{Lee2010c}
Where are the Error functions based on?\\

Form bridge between Geometric Control and Hybrid Control\\
Why Hybrid Control?\\

Parameter Estimation can be done by\\
State Estimation can be done by\\

***************************************\\