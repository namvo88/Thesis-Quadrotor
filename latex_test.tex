\part{TEST PARTS}
%\chapter{Trajectory Generation}\label{ch:trajectory}

***************************************\\
Trajectory is first generated by hand. A simple trajectory that will not push the system to its limits yet\\
Next, trajectory can be generated by solving a \a{QP} via minimum snap generation.

***************************************\\

Trajectory Generation by minimizing Snap Trajectory. QP.\\

\section{Minimum Snap Trajectory Generation}


\section{Conclusion}


%
% First Part
    \part{First Part}

    \chapter{First Real Chapter}

%    This is real chapter for \ac{DUT}, ok? I will explain everything about \gsymb{$\gamma$}{Path Angle}. Next, everything
%    will be explained about the transfer function \lsymb{$H(s)$}{Transfer function}. Also, subscripts and can
    superscripts can be put in the nomenclature \index{nomenclature} list. 
%    \supers{max}{Maximum} 
%    \subs{min}{Minimum} 
    Other things can also
    be added to the nomenclature list of \ac{DUT} 
%    \others{[kts]}{Knots} \others{$^{\circ}$, [deg]}{Degrees}

        \section{First section}

        This is the section. Referring to equations, figures and tables can easily be done by the commands \verb"\eqnref{}",
        \verb"\figref{}" and \verb"\tabref{}".
        \begin{equation}\label{eq:First}
              H(s) = \frac{1}{s+2}
        \end{equation}
        You see? Refer to equations like this \eqnref{eq:First}.
            \subsection{The first subsection}

                \subsubsection[Subsection Short Title]{The first sub-subsection with a very very very long title, but in the table of contents one can only see the short title}

                Nice, ain't it?\index{Nice}

                    \paragraph{A paragraph.}
    \part{Second part}

    \chapter{Second part chapter}
   \begin{figure}
    \caption{this is a very long line to test if the table of
    figures will wrap the line or will continue to go over the
    border  of the page}
    \end{figure}

    \chapter{TEMP second part chapter this is a very long line to test if the table of
    figures will wrap the line or will continue to go over the
    border  of the page}

New chapter gives a full acronym \ac{TU}.

\begin{eqnarray}
% \nonumber to remove numbering (before each equation)
  1 &=& 2\\
  x &=& 5 \\
  y &=& \theta
\end{eqnarray}

    \chapter{Second second part chapter}
    \section{Section}

    This is a test for nomenclature \lsymb{$A(s)$}{Answer function}\\
    \lsymb{$a_f$, $b_f$, $c_f$ and $d_f$}{The variables I am trying to group}\\
    \lsymb{$a_b$}{another variable}\\
\section{Main equations}

\begin{equation}
a=\frac{N}{A}
\end{equation}%

\nomenclature{$a$}{The number of angels per unit area}%
\nomenclature{$N$}{The number of angels per needle point}%
\nomenclature{$A$}{The area of the needle point}%

The equation $\sigma = m a$%
\nomenclature{$\sigma$}{The total mass of angels per unit area}%
\nomenclature{$m$}{The mass of one angel}
follows easily.