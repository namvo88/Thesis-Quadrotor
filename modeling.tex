\chapter{Dynamic Model} \label{ch:model}

%ADD in this section etc
In Section \ref{sec:mod.QRLmod} the model dynamics of the \a{qr}-Load system are obtained by describing the dynamics on nonlinear manifolds, with the concepts of differential geometry. 


%In order to obtain an mathematical model various assumptions are made to simplify the dynamics, this is 
A mathematical model of the system needs to be derived in order to simulate and study the effects of Geometric Control. 
The assumptions that are applied to simplify the model are discussed in Section \ref{sec:mod.assum}.

In Section \ref{sec:mod.geometric} an introduction is given about Geometric Mechanics, which is a modern description of the classical mechanics from the perspective of Differential Geometry. Differential Geometry is a discipline in mathematics that studies manifolds and their geometric properties, using the tools of calculus. Geometric Mechanics is used to model the \a{qr}-Load system, which is described in Section \ref{sec:mod.QRLmod}.

To derive the equations of motions traditional modeling methods often parameterize the rotations in a local coordinate system. Euler angles are commonly used, however these coordinates might result in singularities. Furthermore, there are 24 possible sets of Euler angles and many different conventions are used, which leads to ambiguity. The definition of Euler angles is not unique and a sequence of rotations is not commutative. Therefore, Euler angles are never expressed in terms of the external frame, or in terms of the co-moving rotated body frame, but in a mixture.

In order to avoid these complexities, the dynamics of the \a{qr}-Load system can be globally expressed on the Special Orthogonal Group $SO(3)$, \textit{2-sphere} $ \mathbb{S}^2 $ and Special Euclidean Group $ SE(3) $. This leads to a compact notation of the equations of motion, making the large amount of trigonometric functions unnecessary, that Euler angles normally introduce. 

\section{Modeling Assumptions}\label{sec:mod.assum} 


	%CHECK intermediate frame. nodig?
The \a{qr} model representation is shown in Figure \ref{fig:mod.model}. Three Cartesian coordinate frames are defined:\v{5}
\begin{itemize}
	\setlength\itemsep{.2pt}
	\item The body-fixed reference frame \lsymb{$ \{\mathcal{B}\} $}{Body Frame} (Body Frame)
	\subitem with unit vectors \lsymb{$ \{\mathbf{b}_1,\mathbf{b}_2,\mathbf{b}_3\} $}{Unit vectors along the axes of $ \{\mathcal{B}\} $} along the axes
	\item The ground-fixed reference frame \lsymb{$ \{\mathcal{I} \}$}{Inertial World Frame} (Inertial Frame)
	\subitem with unit vectors \lsymb{$ \{\mathbf{e}_1,\mathbf{e}_2,\mathbf{e}_3\} $}{Unit vectors along the axes of $ \{\mathcal{I}\} $} along the axes								
	\item The intermediary frame \lsymb{$ \{\mathcal{C} \}$}{Intermediary Frame}, ($ \{\mathcal{I} \}$ rotated by the yaw angle $ \psi $) 
	\subitem with unit vectors \lsymb{$ \{\mathbf{c}_1,\mathbf{c}_2,\mathbf{c}_3\} $}{Unit vectors along the axes of $ \{\mathcal{C}\} $} along the axes								
\end{itemize}

\begin{figure}[h!]
	\centering
	\makebox[\textwidth][c]{\includegraphics[width=.5\paperwidth]{./StyleStuff/dcsc.png}}
	\caption{Quadrotor model representation\label{fig:mod.model}}
\end{figure}	

\begin{figure}[h!]
	\centering
	\makebox[\textwidth][c]{\includegraphics[width=.5\paperwidth]{./StyleStuff/dcsc.png}}
	\caption{Quadrotor with Load model representation\label{fig:mod.modelQRL}}
\end{figure}	

The position of the body frame is described by a vector evolving on $ \mathbb{R}^3 $, and is represented with respect to the inertial frame. The orientation, also called attitude, of the body frame with respect to the inertial frame evolves on a nonlinear space, for which several methods exist to describe this, such as \textit{Euler Angles}, quaternions or rotation matrices. 

%CHECK is dit wel nodig?
The complex dynamics of the rotors and their interactions with drag and thrust forces are represented by a simplified model. 
The angular speed \lsymb{$ \omega_i $}{Angular speed of rotor $ i $} of rotor $ i $, for $ i=1,\dots,4 $, generates a force \lsymb{$ F_i $}{Force generated by rotor $ i $} parallel to the direction of the rotor axis of rotor $ i $, given by
\begin{equation}\label{key}
F_i=\left( \frac{K_vK_\tau\sqrt{2\rho A}}{K_t}\omega_i\right)^2=b\omega_i^2 
\end{equation}
where $ K_v,K_t $ are constants related to the motor properties, $ \rho $ is the density of the surrounding air, $ A $ is the area swept out by the rotor, $ K_\tau $ is a constant determined by the blade configuration and parameters, and $ b $ is the thrust factor.\\
The torque around the axis of rotor $ i $, for $ i=1,\dots,4 $, generated due to drag is given by
\begin{equation}\label{key}
M_{i}=\frac{1}{2}R\rho C_DA(\omega_iR)^2=d\omega_i^2
\end{equation}
where $ R $ is the radius of the propeller, $ C_D $ is a dimensionless constant, and $ d $ is the drag constant.

%CHECK directions van de momenten
For given desired total thrust \lsymb{$ f $}{Total thrust. $ f=\sum_{i=1}^{4}F_i $} and total moment \lsymb{$ M $}{Total moment in \BF. $ M=\begin{bmatrix}	M_\phi&M_\theta&M_\psi	\end{bmatrix}^T $}$=\begin{bmatrix}	M_\phi&M_\theta&M_\psi	\end{bmatrix}^T  $, the required rotor speeds can be calculated by solving the following equation
\begin{equation}\label{eq:omega_i}
\begin{bmatrix}
f\\M_\phi\\M_\theta\\M_\psi
\end{bmatrix}=
\begin{bmatrix}
b&b&b&b\\
0&-lb&0&lb\\
lb&0&-lb&0\\
-d&d&-d&d\\
\end{bmatrix}
\begin{bmatrix}
\omega_1^2\\
\omega_2^2\\
\omega_3^2\\
\omega_4^2\\
\end{bmatrix}
\end{equation}
where \lsymb{$ l $}{Distance from the rotor to the QR CoM} is the distance from the rotor to the \a{qr}'s \a{com} and $ M_\phi, M_\theta, M_\psi $ denote the moments around the $ x, y, z $-axis in \BF, resp. 

Table \ref{tab:mod.assumptions} shows the most common assumptions that are used for modeling the \a{qr}, simplifying the complexity of the model.

***************************************\\
Bart: Refer to table 2-2?\\
Nam: Wat bedoel je?

***************************************\\

\begin{table}[h!]
	\centering
	\begin{tabular}{|p{\textwidth}|}
		\hline
		\textbf{Modeling assumptions Quadrotor model}\\
%		\tabitem The rotation of the Earth does not affect the flight of the \a{qr}\\
		\tabitem The structure of the \a{qr} is rigid and symmetric. \\
		\hspace{4mm} Elastic deformations and shock (sudden accelerations) of the \a{qr} are ignored.\\										
		\tabitem The mass distribution of the \a{qr} is symmetrical in the x-y plane.\\
		\tabitem The inertia matrix is time-invariant.\\
		\tabitem Aerodynamic effects acting on the \a{qr} are neglected.\\
		\hspace{4mm} Blade flapping, Turbulence, Ground Effects.\\
		\tabitem The air density around the \a{qr} is constant.\\
%		\hspace{4mm} An indoor environment guarantees the absence of unpredictable disturbances like wind\\ 
%		\hspace{4mm} gusts. The model complexity decreases without modeling the effects of wind.\\ 	
		\tabitem The propellers are rigid $ \Rightarrow $ The thrust produced by rotor $ i $ is parallel to the axis of rotor $ i $.\\
		\tabitem Drag factor \lsymb{$ d $ }{Drag factor} and thrust factor \lsymb{$ b $}{Thrust factor} are approximated by a constant.\\
		\hspace{4mm} Thrust force $ F_i $ and moment \lsymb{$ M_{i} $}{Drag moment generated by each propellor} of each propeller is proportional to the square of \\
		\hspace{4mm} the propeller speed. \\
%		Such that $ F_i = b\omega_i^2$ and $ M_{i} = d\omega_i^2$, where \lsymb{$ \omega_i $}{Angular velocity of rotor $ i $ around its axis, $ i=\{1,2,3,4\} $} is the rotor speed.\\
		\hline
		\textbf{Modeling assumptions Quadrotor-Load model}\\
				\tabitem The cable is modeled as a rigid and massless cable. \\
		\tabitem The cable is connected to a friction-less joint at the origin of the body-fixed. \\
		\tabitem The tension in the cable is considered to be non-zero.\\
		\hspace{4mm} This implies that the QR-Load subsystem, consisting of a separate \a{qr} and Load\\
		\hspace{4mm} in free fall, is disregarded.\\		 
		\tabitem Aerodynamic effects acting on the load are neglected.\\
		\hspace{4mm} reference frame.\\
		\tabitem Assumption \\
		\hspace{4mm} Details Assumption 2\\
		\hline
	\end{tabular}
	\caption{Modeling assumptions}
	\label{tab:mod.assumptions}
\end{table}

%\begin{table}[h!]
%	\centering
%	\begin{tabular}{|p{\textwidth}|}
%		\hline
%		\tabitem The cable is modeled as a rigid and massless cable. \\
%		\tabitem The cable is connected to a friction-less joint at the origin of the body-fixed. \\
%		\tabitem The tension in the cable is considered to be non-zero.\\
%		\hspace{4mm} This implies that the QR-Load subsystem, consisting of a separate \a{qr} and Load\\
%		\hspace{4mm} in free fall, is disregarded.\\		 
%		\tabitem Aerodynamic effects acting on the load are neglected.\\
%		\hspace{4mm} reference frame.\\
%		\tabitem Assumption \\
%		\hspace{4mm} Details Assumption 2\\
%		\hline
%	\end{tabular}
%	\caption{Modeling assumptions Quadrotor-Load model}
%	\label{tab:mod.assumptionsQRL}
%\end{table}


\section{Geometric Mechanics}\label{sec:mod.geometric}
In Geometric Mechanics the configuration space of systems is a \textit{group manifold} instead of a Euclidean space. The kinetic and potential energies are expressed in terms of this configuration space and their tangent spaces. It explores the geometric structure of a Lagrangian- or Hamiltonian system through the concepts of vector calculus, linear algebra, differential geometry, and non-linear control theory. Geometric mechanics provides fundamental insights into the nonlinear system mechanics and yields useful tools for dynamics and control theory.

Euler angles are kinematically singular since the transformation from their time rates of change to the angular velocity vector is not globally defined. Furthermore, when angular errors are large, the difference in Euler angles is no longer a good metric to define the orientation error. Local coordinates often require symbolic computational tools due to complexity of multi-body systems. Hence, the error is rather written as the required rotation to get from the current to a desired orientation. As a result, the equations of motion and the control systems can be developed on a configuration manifold in a coordinate-free, compact, unambiguous manner, while singularities of local parameterization are avoided.

To illustrate the difference in configuration spaces, an example is given of a 2-link arm, where the configuration can be expressed by 2 coordinates, see in Figure \ref{fig:mod.armmanifold}. Figure \ref{fig:mod.armcartesian} represents the configuration space as a Cartesian space, where the same dots represent one of the many identical configurations. This shows that this representation suffers from singularities caused by multiple points in one representation being mapped onto a single point in another representation. Figure \ref{fig:mod.armtorus} shows the configuration space as a geometric shape called a \textit{torus}, a manifold where every configuration is mapped uniquely.
\begin{figure}[h!]
	\centering
	%ADD figure Robot Arm
	%ADD figure cartesian (q1,q2), with the red dot representing sam point
	%ADD figure torus (q1,q2)
	\makebox[.3\textwidth][c]{\subfloat[][\label{fig:mod.arm}]{\includegraphics[width=.3\textwidth]{./StyleStuff/dcsc.png}}}
	\makebox[.3\textwidth][c]{\subfloat[][\label{fig:mod.armcartesian}]{\includegraphics[width=.3\textwidth]{./StyleStuff/dcsc.png}}}
	\makebox[.3\textwidth][c]{\subfloat[][\label{fig:mod.armtorus}]{\includegraphics[width=.3\textwidth]{./StyleStuff/dcsc.png}}}
	\caption{Configuration Space of a 2-link arm\label{fig:mod.armmanifold}}
\end{figure}		

%CHECK
%Mechanics studies the dynamics of physical bodies acting under forces and potential fields. 
%In Lagrangian mechanics, the trajectories are obtained by finding the paths that minimize the integral of a Lagrangian over time, called the action integral. 
%Rigid body dynamics are characterized by Lagrangian/Hamiltonian dynamics. The dynamics of a Lagrangian system has unique geometric properties and these are exploited to obtain Euler-Lagrange equations. The resulting intrinsic form of the Euler-Lagrange equations are more compact than equations expressed in terms of local coordinates.


%ADD PROS CONS
%Problems, singularities with Euler-Angles\\
%Other attitude representations, such as exponential coordinates, quaternions, or Euler
%angles, can also be used following standard descriptions, but each of the representations has a disadvantage
%of introducing an ambiguity or singularity.
%Why charts on $ SO(3) $ \url{https://en.wikipedia.org/wiki/Charts_on_SO(3)}\\


\paragraph{Manifolds}
%ADD  manifolds
The fundamental object of differential geometry a manifold. A manifold is a mathematical space, a collection of points, that locally resembles Euclidean space near each point. Examples are a plane, a ball, a torus and a sphere. Manifolds are important objects in mathematics and physics because they allow more complicated structures to be expressed and understood in terms of the relatively well-understood properties of simpler spaces. In Figure \ref{fig:mod.manifold} is illustrated that each point of an n-dimensional manifold has a neighborhood that is homeomorphic to the n-dimensional Euclidean space, meaning that there is a continuous function describing the relation between these spaces.

\begin{figure}[h!]
	%ADD figure of local space on manifold to cartesian space. Sphere onto R2
	%CHECK \url{https://en.wikipedia.org/wiki/Differentiable_manifold}
	\centering
	\makebox[\textwidth][c]{\includegraphics[width=.45\textwidth]{./StyleStuff/dcsc.png}}
	\caption{A manifold locally resembles a Euclidean space\label{fig:mod.manifold}}
\end{figure}

%ADD 
%An Introduction to Differentiable Manifolds is given by \cite{Boothby2003}. 

A differentiable manifold is a smooth and continuous manifold and is locally similar enough to a linear space to allow to do calculus. One can define directions, tangent spaces, and differentiable functions on such a manifold. Each point of an n-dimensional differentiable manifold has a tangent space, which is an n-dimensional Euclidean space consisting of the tangent vectors of the curves that pass through that point. In Figure \ref{fig:mod.tspace} the manifold $ \mathbb{S}^2 $ represents as a sphere, with a tangent space at point $ x $, denoted by $ T_x\mathbb{S}^2 $. Taking the derivative at a point on a manifold is equivalent to a tangent vector at that point. Meaning that derivatives are conceptually equivalent to an infinitesimally short tangent vector. 
\begin{figure}[h!]
	\centering
	%ADD fig sphere with tangent bundle at x
	\makebox[.49\textwidth][c]{\subfloat[][Representation of a manifold with a tangent space \label{fig:mod.tspace}]{\includegraphics[width=.45\textwidth]{./StyleStuff/dcsc.png}}}
	%ADD fig space with SO(3) and so(3)
	\makebox[.49\textwidth][c]{\subfloat[][Identity map of $ SO(3) $ with Lie Algebra $ \mathfrak{so}(3) $ \label{fig:mod.so3}]{\includegraphics[width=.45\textwidth]{./StyleStuff/dcsc.png}}}
	\caption{\label{fig:}}
\end{figure}		

\subparagraph{Configuration Spaces}
Rotation matrices are used to provide a global representation of the attitude of a rigid body, by mapping a representation of vectors expressed in \BF to a representation expressed in \IF \cite{Chaturvedi2011,Murray1994}. 
The configuration of the \a{qr} attitude is a rotation matrix $ R $ in the Special Orthogonal Group $ SO(3) $ defined as
\begin{equation}\label{eq:SO3}
SO(3) \triangleq \left\lbrace R\in\mathbb{R}^{3\times3}|RR^T=I_{3\times3}, det(R)=1\right\rbrace 
\end{equation}

%CHECK nodig?
%The configuration manifold for combined translational and rotational motion of a rigid body is the special Euclidean group $ SE(3) $, which is a semi-direct product of $ SO(3) $ and $ \mathbb{R}^3 $.

$ SO(3) $ is the group of all rotations about the origin of a 3-D Euclidean space, which preserves the origin, Euclidean distance and orientation.
Every rotation has a unique inverse rotation and the identity map satisfies the definition of a rotation. The elements of \textit{Lie Algebra} $ \mathfrak{so}(3) $, a property associated with $ SO(3) $, are the elements of the tangent space of $ SO(3) $ at the identity element, see Figure \ref{fig:mod.so3}. 
These elements define the relation between the rotation $ R $ and its derivative $ \dot{R} $, such that
\begin{equation}\label{eq:Rdot}
\dot{R} = R\hat{\Omega}
\end{equation}
For $ n\in \mathbb{N} $, $ \mathfrak{so}(3) $ is is the vector space of skew-symmetric matrices in $ \mathbb{R}^{n\times n} $ and defined as
\begin{equation}\label{eq:so3}
\mathfrak{so}(n) \triangleq \left\lbrace S\in \mathbb{R}^{n\times n}|S^T=-S\right\rbrace
\end{equation}

The linear map $ \hat{\cdot}:\mathbb{R}^3\rightarrow\mathfrak{so}(3) $ is an isomorphism between $ \mathbb{R}^3 $ and the set of $ 3\times 3 $ skew symmetric matrices. $ \cdot^\vee:\mathfrak{so}(3)\rightarrow\mathbb{R}^3 $ denotes the inverse isomorphism. The mapping between the body angular velocity vector $ \Omega\in\mathbb{R}^3 $ and  $ \hat{\Omega}\in\mathfrak{so}(3) $ can be written as
\begin{equation}\label{eq:mod.hatOmega}
\hat{\Omega}=\begin{bmatrix}
0&-\Omega_3&\Omega_2\\
\Omega_3&0&-\Omega_1\\
-\Omega_2&\Omega_1&0
\end{bmatrix},
\quad
\begin{bmatrix}
0&-\Omega_3&\Omega_2\\
\Omega_3&0&-\Omega_1\\
-\Omega_2&\Omega_1&0
\end{bmatrix}^\vee = \Omega
\end{equation}

The configuration space of the load is represented on a 2-sphere, defined as
\begin{equation}\label{key}
\mathbb{S}^2 \triangleq \left\lbrace q\in\mathbb{R}^{3}|q\cdot q=1\right\rbrace 
\end{equation}
%\begin{equation}\label{key}
%\mathbb{S}^2 \triangleq \left\lbrace (x_1, x_2, x_3)\in\mathbb{R}^{3}| \parallel (x_1, x_2, x_3)\parallel=1\right\rbrace 
%\end{equation}
%ADD sphere - tangent space - perpendicular : angular velocity
\begin{equation}\label{key}
\dot{q} = \omega\times q
\end{equation}
where $ \omega $ is the angular velocity of the suspended load.

\section{Quadrotor-Load Model}	\label{sec:mod.QRLmod}
The Quadrotor-Load model is shown in Figure \ref{fig:QRLmodel}, where the unit vector \lsymb{$ q$}{Unit vector from \a{qr} to Load} gives the direction from the \a{qr} to the Load expressed in \BF. The focus lies on the subsystem where the cable tension is considered to be non-zero. The position of the \a{qr} and Load are related by
\begin{equation}\label{eq:mod.xQ2xL}
x_Q=x_L-Lq
\end{equation}
where \lsymb{$ x_Q $}{Position of the  of the \a{qr} CoM} is the position of the \a{qr}'s \a{com}, \lsymb{$ x_L $}{Position of the load} is the position of the load, and \lsymb{$ L $}{Length of the cable} is the length of the cable.
%The derived mathematical model is represented by a set of dynamic equations commonly used for rigid body displacements. 

\begin{figure}[h!]
	\centering
	\makebox[\textwidth][c]{\includegraphics[width=.45\textwidth]{./StyleStuff/dcsc.png}}
	\caption{Quadrotor-Load model representation\label{fig:QRLmodel}}
\end{figure}		

Dynamics and optimal control problems for rigid bodies are studied in \cite{Lee2008}, incorporating their geometric features. The focus lies on obtaining geometric properties of the dynamics of rigid bodies, how their configuration can be described and how these geometric properties are utilized in control system analysis and design. 

Considering the properties of the system, the \a{qr} is described as a rigid body with six degrees of freedom, driven by forces and moments. 
%Which means that the motion of a rigid body can be described by a translation of the \acf{com} and a rotation about the \a{com}. 
The configuration of the \a{qr} can be described by the location of its \a{com} and its attitude, which are described in Euclidean space $x_Q\in \mathbb{R}^3 $ and in a nonlinear space $R\in SO(3) $, respectively. 
The configuration of the load can also be described by its location and attitude, described in Euclidean space $x_L\in \mathbb{R}^3 $ and on a two-sphere $ q\in \mathbb{S}^2 $.


%LOAD ATTITUDE DYNAMICS
To develop the Euler-Lagrange equations for mechanical systems that evolve on a Lie group, an approach developed by \cite{Lee2008,Lee2005,Lee2009,Lee2011} is used. 

%ADD Hamilton's principle uitleg?
The basic idea is the variations of the curves that are e	

This approach is based on Hamilton's principle, which states that the evolution of a physical system is a solution of the functional equation given by
\begin{equation}\label{key}
\frac{\delta S}{\delta \mathbf{q}(t)}=0
\end{equation}
%CHECK fact, is that what q defines?
where $ \mathbf{q} $ defines the configuration space. $ S $ is the action integral, defined as
\begin{equation}\label{eq:actionintegral}
S=\int_{t_1}^{t_2}\mathcal{L}dt
\end{equation}
where $\mathcal{L}=\mathcal{T}-\mathcal{U} $ is the Lagrangian of the system, and $\mathcal{T},\mathcal{U}$ are the kinetic and potential energy, respectively. 


%Hamilton's principle requires that the first-order change of  $ \delta S $ is zero for all possible perturbations, meaning that the true path is a stationary point of the action integral, which is defined as
Hamilton's principle of least action states that the path a conservative mechanical system takes between two configurations $ q_1 $ and $ q_2 $ at time $ t_1 $ and $ t_2 $, is the one for which Equation \ref{eq:actionintegral} is a stationary point, resulting in
\begin{equation}\label{eq:HamPr}
\delta S=\int_{t_1}^{t_2}\delta\mathcal{L}dt=0
\end{equation}
where $ \delta\mathcal{L} $ is the variation of the Lagrangian. For systems with non-conservative forces and moments, Equation \ref{eq:HamPr} is extended to
\begin{equation}\label{eq:HamPrNon}
\delta S=\int_{t_1}^{t_2}(\delta W+\delta\mathcal{L})dt=0
\end{equation}
where $ \delta W $ is the virtual work. Equation \ref{eq:HamPrNon} is applied to the QR-Load system, where the configuration manifold is $ \mathbb{R}^3\times \mathbb{S}^2\times SO(3) $. With the following states
\begin{equation}\label{eq:mod.S}
\textbf{x}= \begin{bmatrix}x_L& \dot{x}_L& q& \omega&R&\Omega
\end{bmatrix}^T
\end{equation}


\paragraph{Euler-Lagrange} Equation \ref{eq:HamPr} can be satisfied if the following Euler-Lagrange equation holds
\begin{equation}\label{key}
\frac{\delta \mathcal{L}}{\delta \mathbf{q}}-\frac{d}{dt}\frac{\delta \mathcal{L}}{\delta \dot{\mathbf{q}}}=0
\end{equation}
where the Lagrangian $ \mathcal{L}=\mathcal{T}-\mathcal{U} $.
The kinetic energy for the system is denoted as
\begin{equation}\label{key}
\mathcal{T}=\frac{1}{2}m_Q\dot{x}_Q\cdot\dot{x}_Q+\frac{1}{2}m_L\dot{x}_L\cdot\dot{x}_L+\frac{1}{2}\Omega \cdot J\cdot\Omega
\end{equation}
and the potential energy is denoted as
\begin{equation}\label{key}
\mathcal{U}=m_Qgx_Q\cdot e_3+m_Lgx_L\cdot e_3
\end{equation}
where \lsymb{$ g $}{Gravitation constant} is the gravity constant.
The energy can be rewritten in terms of $ q $ and $ x_L $, by substituting Equation \ref{eq:mod.xQ2xL}, giving
\begin{align}\label{key}
\mathcal{T}&=\frac{1}{2}(m_Q+m_L)\dot{x}_L\cdot\dot{x}_L -m_QL\dot{x}_L\cdot\dot{q} + \frac{1}{2}m_QL^2\dot{q}\cdot\dot{q}+\frac{1}{2}\Omega \cdot J\cdot\Omega\\
\mathcal{U}&=(m_Q+m_L)gx_L\cdot e_3-m_QgLq\cdot e_3
\end{align}
The variations of the $ \mathcal{T} $ and $ \mathcal{U} $ are approximated by a first-order Taylor approximation, which results in
\begin{equation}\label{eq:mod.T}
%ADD
\begin{aligned}
\delta\mathcal{T}&\approx \frac{\partial \mathcal{T}}{\partial\dot{x}_L} \delta\dot{x}_L +\frac{\partial \mathcal{T}}{\partial\dot{q}}\delta\dot{q}+\frac{\partial \mathcal{T}}{\partial\Omega}\delta\Omega\\
&=((m_Q+m_L)\dot{x}_L-m_QL\dot{q})\cdot\delta\dot{x}_L+(-m_QL\dot{x}_L+m_QL^2\dot{q})\cdot\delta\dot{q}+J\Omega\cdot\delta\Omega

\end{aligned}
\end{equation}
%Substituting the constraint in Equation \ref{eq:mod.xQ2xL} and its derivative into Equation \ref{eq:mod.T} results in
%\begin{equation}\label{eq:mod.T2}
%%ADD
%\delta\mathcal{T}=
%\end{equation}

\begin{equation}\label{key}
\begin{aligned}
\delta\mathcal{U}&\approx \frac{\partial \mathcal{U}}{\partial{x}_L} \delta{x}_L +\frac{\partial \mathcal{U}}{\partial{q}}\delta{q}\\
&=(m_Q+m_L)ge_3\cdot\delta x_L-m_QgLe_3\cdot\delta q
\end{aligned}
\end{equation}

The first term of virtual work is obtained from $ f $ acting on the \a{qr} and is given by the following term,
\begin{equation}\label{key}
%ADD
\begin{aligned}
\delta W_1&=fRe_3\cdot \sum_{j=1}^{3}\frac{\partial x_Q}{\partial \mathbf{q}_j}\delta \mathbf{q}_j\\
&=fRe_3\cdot(\delta x_L-L\delta q)
\end{aligned}
\end{equation}
where $ \mathbf{q_j}={x_L,q,R} $ and $ x_Q $ is substituted by Equation \ref{eq:mod.xQ2xL}
The second term of virtual work is obtained from $ M $ acting on the \a{qr}. This gives the following term
\begin{equation}\label{key}
\begin{aligned}
\delta W_2&=M\cdot \sum_{j=1}^{3}\frac{\partial\Omega}{\partial \mathbf{\dot{q}}_j}\delta \mathbf{\dot{q}}_j\\
&=M\cdot(R^T\delta R)

\end{aligned}\end{equation}

The variations in energy and the virtual work can be substituted into Equation \ref{eq:mod.S}.

%ADD explain / described in Bullo
Equation \ref{eq:Sfinal} is a function of variations on manifolds, where $ \delta R $ is a variation on $ SO(3) $ and $ \delta q $ is a variation on $ \mathbb{S}^2 $. 
The so called infinitesimal variations required to solve this equation are described as \cite{Bullo2005,Sreenath2013c}
\begin{equation}\label{key}
\begin{aligned}
\delta q&=\xi\times q \in T_q\mathbb{S}^2\text{, where }\xi\in\mathbb{R}^3,\xi\cdot q=0\\
\delta \dot{q}&=\\
\delta R&=R\hat{\eta}\in T_RSO(3)\text{, where } \eta\in\mathbb{R}^3,\hat{\eta}\in\mathfrak{so}(3)\\
\delta \dot{R}&=\\
\delta \hat{\Omega}&=\\
\end{aligned}
\end{equation}

Substituting the variations in energy and the variations in 
\begin{equation}\label{eq:Sfinal}
%ADD
\begin{aligned}
\delta S &= \int_{t_1}^{t_2}(\delta W_1+\delta W_2+\delta\mathcal{T}-\delta\mathcal{U})dt\\
&=\\
&=
\end{aligned}
\end{equation}



%CHECK nodig?
%***************************************\\
%Rigid Body Attitude Dynamics evolve on $ SE(3) $.
%\begin{align}\label{eq:eomrigidbody}
%%CHECK waar komt deze equation vandaan?
%J\dot{\Omega}+\Omega\times J\Omega &= mg\rho\times R^Te_3+u\\ 
%\dot{R} &= R\hat{\Omega}
%\end{align}
%***************************************\\

%CHECK standard newton euler
%***************************************\\
%The equations of motion for a rigid body with configuration $ SE(3) $ are given by the \textit{Newton-Euler equations} \cite{Murray1994}:
%\begin{equation}\label{key}
%\begin{bmatrix}
%	mI&0\\
%	0&\mathcal{I}
%\end{bmatrix}
%\begin{bmatrix}
%	\dot{v}^b\\
%	\dot{\omega}^b
%\end{bmatrix}+
%\begin{bmatrix}
%	\omega^b\times mv^b\\
%	\omega^b\times\mathcal{I}\omega^b
%\end{bmatrix}=F^b
%\end{equation}
%where $ m $ is the mass of the body, $ \mathcal{I} $ is the inertia tensor, and $ V^b=(v^b,\omega^b) $ and $ F^b $ represent the instantaneous body velocity and applied body wrench.
%***************************************\\

%ADD equations of motion
%ADD From .... follows
The \a{qr} attitude kinematics equation is given by
\begin{equation}\label{key}
\begin{aligned}
\dot{x}_L&=\\
(m_Q+m_L)(\dot{v}_L+ge_3)&=\\
\dot{q}&=\\
m_QL\dot{\omega}&=\\
\dot{R}&=R\hat{\Omega}\\
J\dot{\Omega}+\Omega\times J\Omega&=
\end{aligned}
\end{equation}
%\begin{equation}\label{key}
%\hat{\Omega}=\begin{bmatrix}
%0&-\Omega_3&\Omega_2\\
%\Omega_3&0&-\Omega_1\\
%-\Omega_2&\Omega_1&0
%\end{bmatrix}
%\end{equation}


\section{Classical Modeling}\label{sec:mod.clas}

%CHECK waarom dit? Per se nodig om LQR uit te leggen?
This section describes the derivation of the model by using classical modeling techniques.\\
NEEDED??


%CHECK is deze reference nog relevant?
%Reference \ref{app:model}

When assuming small angle maneuvers, \textit{Euler-angles} can be used to locally parameterize the orientation of the body-fixed reference coordinate frame with respect to the inertial reference coordinate frame. Simple linear controllers are often based on a linearized dynamical model, applying this small angles assumption. 

From Newton's law follows
\begin{equation}\label{eq:newton}
\begin{aligned}
\dot{x}_Q &= v_Q\\
m_Q\dot{v}_Q &=fRe_3-m_Qge_3-Tq\\
\dot{x}_L &= v_L\\
m_L\dot{v}_L &=-m_Lge_3+Tq
\end{aligned}
\end{equation}
%which gives the following equation, derived in Section \ref{sec.app:loaddyn},
%\begin{equation}\label{key}
%%CHECK whether equation is correct
%(m_Q+m_L)(\dot{v}_L+ge_3)=fRe_3-m_QL\ddot{q}
%\end{equation}


Because the Euler-Angles are used, a function is required that maps a vector of the Z-X-Y Euler angles to its rotation matrix $ R\in SO(3) $, which is denoted as \cite{Mahony2012}
\begin{equation}\label{key}
R_{312}({\phi},{\theta},{\psi})=\begin{bmatrix}
c_{\psi}c_{\theta}-s_{\phi}s_{\psi}s_{\theta}&-c_{\phi}s_{\psi}&c_{\psi}s_{\theta}+c_{\theta}s_{\phi}s_{\psi}\\
c_{\theta}s_{\psi}+c_{\psi}s_{\phi}s_{\theta}&c_{\phi}c_{\psi}&s_{\psi}s_{\theta}-c_{\psi}c_{\theta}s_{\phi}\\
-c_{\phi}s_{\theta}&s_{\phi}&c_{\phi}c_{\theta}
\end{bmatrix}
\end{equation}
The Z-X-Y Euler angles to model the rotation can be seen in Figure \ref{fig:mod.modelQRtrad}. The first rotation by yaw angle $ \psi $ is around the z-axis of \IF. Next is the rotation by roll angle $ \phi $, and the last rotation is by pitch angle $ \theta $.
\begin{figure}[h!]
	\centering
	\makebox[\textwidth][c]{\includegraphics[width=.45\textwidth]{./StyleStuff/qrmodelppt.png}}
	\caption{Quadrotor-Load model representation\label{fig:mod.modelQRtrad}}
\end{figure}

The unit vector $ q $ from the \a{qr} to the load is represented in \BF. Define $ \phi_L $ as the yaw-rotation of the load around the z-axis of \BF and $ \theta_L $ as the angle between the cable and the z-axis of \BF, see Figure \ref{fig:mod.modelQRLtrad}.
\begin{figure}[h!]
	%ADD figure with angles phiL and theta L
	\centering
	\makebox[\textwidth][c]{\includegraphics[width=.45\textwidth]{./StyleStuff/dcsc.png}}
	\caption{\label{fig:mod.modelQRLtrad}}
\end{figure}	

\begin{equation}\label{eq:q}
%		q=-R_{\psi_L}R_{\theta_L}e_3=\begin{bmatrix}
%		s_{\theta_L}c_{\psi_L}\\s_{\theta_L}s_{\psi_L}\\-c_{\theta_L}
%		\end{bmatrix}
%\begin{align}
q=\begin{bmatrix}
s_{\theta_L}c_{\phi_L}\\
s_{\theta_L}s_{\phi_L}\\
c_{\theta_L}
\end{bmatrix}
%\dot{q}&=\begin{bmatrix}
%s_{\theta_L}c_{\psi_L}\\
%s_{\theta_L}s_{\psi_L}\\
%c_{\theta_L}
%\end{bmatrix}\\
%\end{align}
\end{equation}  
Differentiating Equation (\ref{eq:mod.xQ2xL}) and (\ref{eq:q}) gives
\begin{equation}\label{key}
\begin{aligned}
\ddot{x}_L&=\ddot{x}_Q-\ddot{q}L\\
\ddot{q}&=\begin{bmatrix}
\ddot{\theta}_Lc_{\theta_L}c_{\phi_L}-\ddot{\phi}_Ls_{\theta_L}s_{\phi_L}-\dot{\phi}_L^2s_{\theta_L}c_{\phi_L}-\dot{\theta}_L^2s_{\theta_L}c_{\phi_L}-2\dot{\theta}_L\dot{\phi}_Lc_{\theta_L}s_{\phi_L}\\
\ddot{\theta}_Lc_{\theta_L}s_{\phi_L}+\ddot{\phi}_Ls_{\theta_L}c_{\phi_L}-\dot{\phi}_L^2s_{\theta_L}s_{\phi_L}-\dot{\theta}_L^2s_{\theta_L}s_{\phi_L}+2\dot{\theta}_L\dot{\phi}_Lc_{\theta_L}c_{\phi_L}\\
-\ddot{\theta}_Ls_{\theta_L}-\dot{\theta}_L^2 c_{\theta_L}\\
\end{bmatrix}
\end{aligned}
\end{equation}

\begin{equation}\label{key}
\begin{aligned}
\ddot{x}_Q&=\frac{1}{m_Q}(f(c_{\psi}s_{\theta}+c_{\theta}s_{\phi}s_{\psi})-Ts_{\theta_L}c_{\psi_L})\\
\ddot{y}_Q&=\frac{1}{m_Q}(f(s_{\psi}s_{\theta}-c_{\psi}c_{\theta}s_{\phi})-Ts_{\theta_L}s_{\psi_L})\\
\ddot{z}_Q&=\frac{1}{m_Q}(f(c_{\phi}c_{\theta})-Tc_{\theta_L})-g\\
\end{aligned}
\end{equation}

\begin{align}\label{key}
%CHECK wat is hier de bedoeling van? Checken in Garcia of literatuur?
	\ddot{\psi}&=\tilde{\tau}_{\psi}\\
\ddot{\theta}&=\tilde{\tau}_{\theta}\\
\ddot{\phi} &=\tilde{\tau}_{\phi}
\end{align}

\section{Stability Analysis}
Lyapunov Analysis on SO3 x R3 and S2 x R3
Closed-loop full-attitude dynamics evolve on the non- Euclidean manifold SO3 x R3, while closed-loop re-
duced-attitude dynamics evolve on the non-Euclidean mani- fold S2 x R3. Since these manifolds are locally Euclidean, local stability properties of a closed-loop equilibrium solution can be assessed using standard Lyapunov methods. In ad- dition, the LaSalle invariance result and related Lyapunov results apply to closed-loop vector fields defined on these
manifolds. However, since the manifolds SO132 and S2 are compact, the radial unboundedness assumption cannot be satisfied; consequently, global asymptotic stability cannot fol- low from a Lyapunov analysis on Euclidean spaces [40], and therefore must be analyzed in alternative ways [19]–[23].\cite[p.43]{Chaturvedi2011}

%HOeft waarschijnlijk niet eens een section te zijn, kan kort en bondig
\section{Summary}

***************************************\\
Compact, unambiguous, globally defined, 

Pro/Cons of Classical Modeling Techniques vs Geometric Modeling\\

Linearized model/State Space model vs. Geometric modeling\\


Geometric Mechanics/Lie Groups/Lie Algebra is used in order to represent the dynamics of the system onto the nonlinear configuration manifold $ SE(3) $\\
Advantage of this method is\\
Enables to model on \\
That type of control is discussed in the next chapter


***************************************\\