% arara: pdflatex
% arara: nomencl
% arara: pdflatex
\documentclass[a4paper,11pt]{mscThesis}

%
%========================== Packages ======================================
		\usepackage{lipsum} 
		\usepackage{outlines}
		\usepackage{amsmath}
		\usepackage{amssymb}
		\usepackage{tabularx}
		\usepackage{graphicx}
		\usepackage{subfig}
		\usepackage{hhline}
		\usepackage{epstopdf}
		\usepackage{url}
		
		% Acronyms
		\usepackage{acronym}
		%Nomenclature
		\usepackage{nomencl}
		\makenomenclature		
		%Index package
		\usepackage{imakeidx}		
		\makeindex
		
		\renewcommand{\a}[1]{\acs{#1}} %Fastkey \acs{[1]}
		\renewcommand{\v}[1]{\vspace{-#1mm}} 
		\newcommand{\IF}{$\left\lbrace\mathcal{I}\right\rbrace$ }
		\newcommand{\BF}{$\left\lbrace\mathcal{B}\right\rbrace$ }	
		
		%Tabitems
		\usepackage{booktabs}% http://ctan.org/pkg/booktabs
		\newcommand{\tabitem}{~~\llap{\textbullet}~~}
		
		% Bibliography
		\usepackage[utf8]{inputenc}

%================================================================		

\mscDepartment{Delft Center for Systems and Control}
%{Systems and Control}
\mscProgram{Mechanical Engineering} \mscFaculty{Mechanical,
Maritime and Materials Engineering} \mscName{N.N. Vo}
\mscDate{\today} \mscTitle{Geometric Control of a Quadrotor with a Suspended Load} \mscSubTitle{Subtitle}
\mscKeyWords{thesis, msc, subject}
%
%use the next line if you want a picture on the title page
\mscTitlePagePicture{example_titlefig}
%Note: the file example_titlefig.eps is just for illustration. Do not use this for your own thesis

%% --------------------------------------------------------------------
%% THIRD PARTY OPTIONS

%use the next line if you want a text to acknowledge (a) third party/parties on the copyright page
\mscThirdPartyText{The work in Master of Science Thesis was supported by Alten. Their cooperation is hereby gratefully acknowledged.}

%use the next line if you want the logo of a third party on the copyright page
%NOTE: on the title page only the TUDelft and DCSC logo's are permitted. These are automatically created

\mscThirdPartyLogo{./StyleStuff/ALTEN.png}
%Note: the file examplelogo.eps is just for illustration. Do not use this for your own thesis

%% --------------------------------------------------------------------


\mscReaderOne{dr.ir. T. Keviczky}
\mscReaderTwo{ir. B. van Vliet}
%\mscReaderThree{ir. My. Readertwo}
%\mscReaderFour{}
%
\setThesisInfo
%

\begin{document}
%
%============================= Front matter ========================================
\frontmatter %
%
% Make a hell of a lot of title pages
 \maketitle
%
% Abstract
\nonumchap{Abstract}

A Quadrotor is a type of Unmanned Aerial Vehicle that has received an increasing amount of attention recently with many applications including search and rescue, surveillance, supply of food and medicines as disaster relief and object manipulation in construction and transportation.

An interesting control problem is the Load Position Tracking of a cable suspended load. The system is highly nonlinear and under-actuated. The load cannot be actuated directly and has a natural swing at the end of each Quadrotor movement. A Nonlinear Geometric Control approach allows differential geometric techniques to be applied to systems control, which can be defined on a smooth nonlinear configuration space. This creates a coordinate-free dynamic model, while avoiding the problem of singularities on local charts. 


***************************************\\
Intro about GC.. 
Reasons to consider GC..

***************************************\\
Where simple linear control methods are restricted to small angle movements, nonlinear control methods allow more aggressive and faster movements.  The goal of the project is to investigate the effects on load position tracking performance when the system is modeled and controlled via a Nonlinear Geometric Control approach.


***************************************\\
The Quadrotor-Load system is modeled in a compact and coordinate-free fashion which allows the inherent geometric properties of the system to be controlled. 
%BOVENSTAAND IS VAAG?


The main goal of this thesis is to research the effects on a cable-suspended load transportation using quadrotors, by involving complex or aggressive maneuvering through implementation of Non-Linear Geometric Control.
Where linear control methods are restricted to small angle movement, non-linear control methods allow more aggressive movements. 

%
%		The main goal of this literature study is to review researches that have been done regarding the possibilities of manipulation and transportation of cable-suspended loads using quadrotors, possibly involving complex or aggressive maneuvering. Where linear control methods are restricted to small angle movement, non-linear control methods allow more aggressive movements, but are subject to complex 
%		***************************************\\
%		What is complex? If that results in super easy feedback loop, why not consider it?
%		
%		***************************************\\
%		mathematics and are more computationally intensive and less robust. 


%***************************************\\
%In the considered research papers, different modeling methods and control techniques are applied which are suitable for various specific applications. The models, derivation and underlying assumptions are studied and explained in detail. 		
%
%Super general. Can be placed in every paper.
%
%***************************************\\

Furthermore, the studied control techniques are explained and their advantages are addressed. Several trajectory generation approaches and the related optimization techniques are studied. Their applications, with different purposes such as obstacle avoidance, time-optimal and swing-free trajectory planning are explained. The survey is concluded with a discussion about finding a suitable
***************************************\\
Define suitable

***************************************\\
control design to achieve the quadrotor-assisted task involving manipulation of a cable-suspended load.

***************************************\\


		
\cleardoublepage
%
% Acknowledgements
    \nonumchap{Acknowledgements}%
 %NOG NIET AF
    I would like to thank my supervisors \mscreaderone\ from DCSC and \mscreadertwo\ from Alten Nederland B.V. for their assistance during the writing
    of this thesis. I would also \ldots 
    \vspace*{15mm}

    \noindent
    Delft, University of Technology \hfill \mscname\\
    \mscdate

%
% table of contents, (\toc of \toclof of \tocloflot )
    \tocloflot
%
    
    \cleardoublepage%
%
%============================= Main matter =========================================
%
\mainmatter
%
% Introduction
%\include{examples}
%FROM LITERATURE
%\chapter*{Plan of approach}

Geometric Control looks most promising for attitude tracking of the QR and includes the ability to maneuver aggressively. To build almost-globally attractive controllers, stability of the error dynamics must be proven in order to guarantee the overall stability of the complete dynamics. Based on Geometric Control, Hybrid Control considers the different dynamics of a quadrotor-load system and makes this is a potential candidate for the position tracking of the load and attitude tracking of the quadrotor.

Minimizing Snap Trajectory Generation, where a Quadratic Programming problem is solved, looks most promising in terms of computational load and feasibility to generate an aggressive, yet smooth trajectory for both quadrotor and quadrotor-load systems. Obstacle avoidance constraints can be added by using integer variables and solving a Mixed Integer Quadratic Programming problem.

The goal is to make a control design for the QR-Load system based on Geometric Modeling and Geometric Control, which is capable of tracking a load trajectory, while maintaining the attitude control of the QR. The objective is divided into subproblems and the following approach is proposed to achieve this objective.

The feasibility of the nonlinear geometric control design has to be shown for the QR system, before involving the dynamics of a suspended load. Next, it must be proven that the proposed Hybrid Control of the QR-Load system is able to track a load trajectory, allowing switches between the QR-Load subsystem and QR subsystem. By defining a trigonometric function, the trajectory tracking can be tested in a controlled manner. A trajectory can be optimized by minimizing an objective function of the fourth and sixth derivative of position w.r.t. time, which is able to take various constraints into account by using the properties of differential flatness.
\section{QR System}
\begin{outline}
\1 Modeling QR system, using geometric algebra and Lie algebra
\begin{align}
	\dot{x}&=v\\
	m\dot{v}&=mge_3-fRe_3\\
	\dot{R}&=R\hat{\Omega}\\
	M&=J\dot{\Omega}+\Omega\times J\Omega
\end{align} 
\1 Simulation using:
\2 Simulink
\2 MATLAB
\2 MATLAB Robotics Toolbox \footnote{\url{http://www.petercorke.com/RTB/robot.pdf}}												
\1 Reproduce Geometric Position Control							
\1 Reproduce Geometric Attitude Control	
\1 Reproduce Trajectory Generation
\1 Incorporate effects of model- and parameter uncertainties
\begin{align}
\dot{x}&=v\\
m\dot{v}&=mge_3-fRe_3+\Delta_x\\
\dot{R}&=R\hat{\Omega}\\
M+\Delta_R&=J\dot{\Omega}+\Omega\times J\Omega
\end{align} 
where $ \Delta_x $ and $ \Delta_R \in\mathbb{R}^3$ denote unstructured, but fixed uncertainties in the translational and rotational dynamics.
\1 Incorporate effects of sensor noise and filters							
\1 Simulation including effects of implementation
\end{outline}						

If the feasibility of the nonlinear geometric control design is shown, the QR-Load system is considered by incorporating the dynamic effects of the load. 
For the QR-Load system, the same approach as in the previous section is used. Simulation must prove the feasibility of the theories.
\section{QR-Load System}
\begin{outline}
	\1 Modeling of the QR-Load system, using geometric algebra and Lie algebra.
	For the QR-Load Subsystem, with non-zero cable tension
	\begin{align}
	\dot{x}_L&=v_L\\
	(m+m_L)(\dot{v}_L+ge_3)&=(q\cdot fRe_3-ml(\dot{q}\cdot\dot{q}))q\\
	\dot{q}&=\omega\times q\\
	ml\dot{\omega}&=-q\times fRe_3\\
	\dot{R}&=R\hat{\Omega}\\
	M&=J\dot{\Omega}+\Omega\times J\Omega
	\end{align} 

	For the QR Subsystem, with zero cable tension
	\begin{align}
	\dot{x}_L&=v_L\\
	m_L(\dot{v}_L+ge_3)&=0\\
	\dot{x}&=v\\
	m\dot{v}&=mge_3-fRe_3\\
	\dot{R}&=R\hat{\Omega}\\
	M&=J\dot{\Omega}+\Omega\times J\Omega
	\end{align} 
	
	\1 Hybrid Control Design	
	\begin{align}
	\Sigma_n&=\begin{cases}
	\dot{X}_n=f_n(X_n)+g_n(X_n)u, & X_n\notin\mathcal{S}_z\\
	X^+_z=\Delta_{n\rightarrow z}(X_n^-), & X_n\in\mathcal{S}_z
	\end{cases}\\	
	\Sigma_z&=\begin{cases}
	\dot{X}_z=f_z(X_z)+g_z(X_z)u, & 	X_z\notin\mathcal{S}_n\\
	X_n^+=\Delta_{z\rightarrow n}(X_z^-),& 	X_z\in\mathcal{S}_n
	\end{cases}	
	\end{align}
	where
	\begin{align}
	X_n&=\left\{x_L,q,R,v_L,\omega,\Omega\right\}\\
	X_z&=\left\{x_L,x_Q,R,v_L,v_Q,\Omega\right\}\\
	u&=\left\{f,M\right\}\\
	&\text{Where the guards are defined as}\\
	\mathcal{S}_n&=\left\{X_z|\parallel x_Q-x_L\parallel\equiv l,\frac{d}{dt}\parallel x_Q-x_L\parallel>0\right\}\\
	\mathcal{S}_z&=\left\{X_n|T\equiv0\right\}\\
	&\text{Where the tension is defined as}\\
	T&:=\parallel m_L(\ddot{x}_L+ge_3)\parallel
	\end{align}
	where $ \Delta_{n\rightarrow z} $ is an identity map, and $ \Delta_{z\rightarrow n} $ is modeled as an inelastic collision of two objects, that ensures $ \dot{x}_Q^+-\dot{x}_L^+=0 $.

	\2 Test for the planar 2-D case 
	\2 Test for the full 3-D case	
	\3 Reproduce Geometric Attitude Control	for QR	
	\3 Reproduce Geometric Attitude Control	for Load	
	\3 Reproduce Geometric Position Control	for Load						
	\3 Reproduce Trajectory Generation for Load
		
\section{Trajectory Generation}
Minimizing the sixth-derivative of the load position ensures minimum snap motion for the QR. The following must be solved,
\begin{equation}\label{key}
\min \int_{t_0}^{t_1}\parallel\frac{d^kx_i}{dt^k}\parallel^2dt
\end{equation}
\1 \textbf{Obstacle avoidance}
\2 Trajectory generation, as done by \cite{Tang2015}
\end{outline}				


\section{QR System Implementation}		
\begin{outline}
\1 Set up communication with Optitrack, Paparazzi's Nat2Ivy
\1 Set up manual control and use known \a{uav} parameters for modeling
\1 Test Geometric Position Control with existing Attitude Control 
\1 Test Geometric Attitude Control				

\1 \textbf{Optional; Improve Model}
\2 System identification with \textbf{\a{indi}}\cite{Smeur2015,Smeur2016}
\2 Evaluate the validity of identified system
\2 Dynamic modeling of QR system				

\1 \textbf{Optional; Off-board Control }\\
This will allow off-board communication and calculations				
\2 Set up former framework as done by \cite{PraveenThesis}
\end{outline}

    \chapter{Introduction} \label{chap:intro}

    This is a \LaTeX thesis and this is Chapter\ \ref{chap:intro}.
    If you want to know more about \LaTeX you better read
    \cite{texbook}.\index{LaTeX}  It contains an acronym
    of the \ac{DUT}. The \ac{DUT} is our University. 
    \cleardoublepage
    
    ***************************************\\
    Aim and Motivation\\
    State of the art methods\\
    Research and Engineering goals\\
    Organization of the report
    
    ***************************************\\
\chapter{Dynamic Model} \label{ch:model}

%ADD in this section etc
In Section \ref{sec:mod.QRLmod} the model dynamics of the \a{qr}-Load system are obtained by describing the dynamics on nonlinear manifolds, with the concepts of differential geometry. 


%In order to obtain an mathematical model various assumptions are made to simplify the dynamics, this is 
A mathematical model of the system needs to be derived in order to simulate and study the effects of Geometric Control. 
The assumptions that are applied to simplify the model are discussed in Section \ref{sec:mod.assum}.

In Section \ref{sec:mod.geometric} an introduction is given about Geometric Mechanics, which is a modern description of the classical mechanics from the perspective of Differential Geometry. Differential Geometry is a discipline in mathematics that studies manifolds and their geometric properties, using the tools of calculus. Geometric Mechanics is used to model the \a{qr}-Load system, which is described in Section \ref{sec:mod.QRLmod}.

To derive the equations of motions traditional modeling methods often parameterize the rotations in a local coordinate system. Euler angles are commonly used, however these coordinates might result in singularities. Furthermore, there are 24 possible sets of Euler angles and many different conventions are used, which leads to ambiguity. The definition of Euler angles is not unique and a sequence of rotations is not commutative. Therefore, Euler angles are never expressed in terms of the external frame, or in terms of the co-moving rotated body frame, but in a mixture.

In order to avoid these complexities, the dynamics of the \a{qr}-Load system can be globally expressed on the Special Orthogonal Group $SO(3)$, \textit{2-sphere} $ \mathbb{S}^2 $ and Special Euclidean Group $ SE(3) $. This leads to a compact notation of the equations of motion, making the large amount of trigonometric functions unnecessary, that Euler angles normally introduce. 

\section{Modeling Assumptions}\label{sec:mod.assum} 


	%CHECK intermediate frame. nodig?
The \a{qr} model representation is shown in Figure \ref{fig:mod.model}. Three Cartesian coordinate frames are defined:\v{5}
\begin{itemize}
	\setlength\itemsep{.2pt}
	\item The body-fixed reference frame \lsymb{$ \{\mathcal{B}\} $}{Body Frame} (Body Frame)
	\subitem with unit vectors \lsymb{$ \{\mathbf{b}_1,\mathbf{b}_2,\mathbf{b}_3\} $}{Unit vectors along the axes of $ \{\mathcal{B}\} $} along the axes
	\item The ground-fixed reference frame \lsymb{$ \{\mathcal{I} \}$}{Inertial World Frame} (Inertial Frame)
	\subitem with unit vectors \lsymb{$ \{\mathbf{e}_1,\mathbf{e}_2,\mathbf{e}_3\} $}{Unit vectors along the axes of $ \{\mathcal{I}\} $} along the axes								
	\item The intermediary frame \lsymb{$ \{\mathcal{C} \}$}{Intermediary Frame}, ($ \{\mathcal{I} \}$ rotated by the yaw angle $ \psi $) 
	\subitem with unit vectors \lsymb{$ \{\mathbf{c}_1,\mathbf{c}_2,\mathbf{c}_3\} $}{Unit vectors along the axes of $ \{\mathcal{C}\} $} along the axes								
\end{itemize}

\begin{figure}[h!]
	\centering
	\makebox[\textwidth][c]{\includegraphics[width=.5\paperwidth]{./StyleStuff/dcsc.png}}
	\caption{Quadrotor model representation\label{fig:mod.model}}
\end{figure}	

\begin{figure}[h!]
	\centering
	\makebox[\textwidth][c]{\includegraphics[width=.5\paperwidth]{./StyleStuff/dcsc.png}}
	\caption{Quadrotor with Load model representation\label{fig:mod.modelQRL}}
\end{figure}	

The position of the body frame is described by a vector evolving on $ \mathbb{R}^3 $, and is represented with respect to the inertial frame. The orientation, also called attitude, of the body frame with respect to the inertial frame evolves on a nonlinear space, for which several methods exist to describe this, such as \textit{Euler Angles}, quaternions or rotation matrices. 

%CHECK is dit wel nodig?
The complex dynamics of the rotors and their interactions with drag and thrust forces are represented by a simplified model. 
The angular speed \lsymb{$ \omega_i $}{Angular speed of rotor $ i $} of rotor $ i $, for $ i=1,\dots,4 $, generates a force \lsymb{$ F_i $}{Force generated by rotor $ i $} parallel to the direction of the rotor axis of rotor $ i $, given by
\begin{equation}\label{key}
F_i=\left( \frac{K_vK_\tau\sqrt{2\rho A}}{K_t}\omega_i\right)^2=b\omega_i^2 
\end{equation}
where $ K_v,K_t $ are constants related to the motor properties, $ \rho $ is the density of the surrounding air, $ A $ is the area swept out by the rotor, $ K_\tau $ is a constant determined by the blade configuration and parameters, and $ b $ is the thrust factor.\\
The torque around the axis of rotor $ i $, for $ i=1,\dots,4 $, generated due to drag is given by
\begin{equation}\label{key}
M_{i}=\frac{1}{2}R\rho C_DA(\omega_iR)^2=d\omega_i^2
\end{equation}
where $ R $ is the radius of the propeller, $ C_D $ is a dimensionless constant, and $ d $ is the drag constant.

%CHECK directions van de momenten
For given desired total thrust \lsymb{$ f $}{Total thrust. $ f=\sum_{i=1}^{4}F_i $} and total moment \lsymb{$ M $}{Total moment in \BF. $ M=\begin{bmatrix}	M_\phi&M_\theta&M_\psi	\end{bmatrix}^T $}$=\begin{bmatrix}	M_\phi&M_\theta&M_\psi	\end{bmatrix}^T  $, the required rotor speeds can be calculated by solving the following equation
\begin{equation}\label{eq:omega_i}
\begin{bmatrix}
f\\M_\phi\\M_\theta\\M_\psi
\end{bmatrix}=
\begin{bmatrix}
b&b&b&b\\
0&-lb&0&lb\\
lb&0&-lb&0\\
-d&d&-d&d\\
\end{bmatrix}
\begin{bmatrix}
\omega_1^2\\
\omega_2^2\\
\omega_3^2\\
\omega_4^2\\
\end{bmatrix}
\end{equation}
where \lsymb{$ l $}{Distance from the rotor to the QR CoM} is the distance from the rotor to the \a{qr}'s \a{com} and $ M_\phi, M_\theta, M_\psi $ denote the moments around the $ x, y, z $-axis in \BF, resp. 

Table \ref{tab:mod.assumptions} shows the most common assumptions that are used for modeling the \a{qr}, simplifying the complexity of the model.

***************************************\\
Bart: Refer to table 2-2?\\
Nam: Wat bedoel je?

***************************************\\

\begin{table}[h!]
	\centering
	\begin{tabular}{|p{\textwidth}|}
		\hline
		\textbf{Modeling assumptions Quadrotor model}\\
%		\tabitem The rotation of the Earth does not affect the flight of the \a{qr}\\
		\tabitem The structure of the \a{qr} is rigid and symmetric. \\
		\hspace{4mm} Elastic deformations and shock (sudden accelerations) of the \a{qr} are ignored.\\										
		\tabitem The mass distribution of the \a{qr} is symmetrical in the x-y plane.\\
		\tabitem The inertia matrix is time-invariant.\\
		\tabitem Aerodynamic effects acting on the \a{qr} are neglected.\\
		\hspace{4mm} Blade flapping, Turbulence, Ground Effects.\\
		\tabitem The air density around the \a{qr} is constant.\\
%		\hspace{4mm} An indoor environment guarantees the absence of unpredictable disturbances like wind\\ 
%		\hspace{4mm} gusts. The model complexity decreases without modeling the effects of wind.\\ 	
		\tabitem The propellers are rigid $ \Rightarrow $ The thrust produced by rotor $ i $ is parallel to the axis of rotor $ i $.\\
		\tabitem Drag factor \lsymb{$ d $ }{Drag factor} and thrust factor \lsymb{$ b $}{Thrust factor} are approximated by a constant.\\
		\hspace{4mm} Thrust force $ F_i $ and moment \lsymb{$ M_{i} $}{Drag moment generated by each propellor} of each propeller is proportional to the square of \\
		\hspace{4mm} the propeller speed. \\
%		Such that $ F_i = b\omega_i^2$ and $ M_{i} = d\omega_i^2$, where \lsymb{$ \omega_i $}{Angular velocity of rotor $ i $ around its axis, $ i=\{1,2,3,4\} $} is the rotor speed.\\
		\hline
		\textbf{Modeling assumptions Quadrotor-Load model}\\
				\tabitem The cable is modeled as a rigid and massless cable. \\
		\tabitem The cable is connected to a friction-less joint at the origin of the body-fixed. \\
		\tabitem The tension in the cable is considered to be non-zero.\\
		\hspace{4mm} This implies that the QR-Load subsystem, consisting of a separate \a{qr} and Load\\
		\hspace{4mm} in free fall, is disregarded.\\		 
		\tabitem Aerodynamic effects acting on the load are neglected.\\
		\hspace{4mm} reference frame.\\
		\tabitem Assumption \\
		\hspace{4mm} Details Assumption 2\\
		\hline
	\end{tabular}
	\caption{Modeling assumptions}
	\label{tab:mod.assumptions}
\end{table}

%\begin{table}[h!]
%	\centering
%	\begin{tabular}{|p{\textwidth}|}
%		\hline
%		\tabitem The cable is modeled as a rigid and massless cable. \\
%		\tabitem The cable is connected to a friction-less joint at the origin of the body-fixed. \\
%		\tabitem The tension in the cable is considered to be non-zero.\\
%		\hspace{4mm} This implies that the QR-Load subsystem, consisting of a separate \a{qr} and Load\\
%		\hspace{4mm} in free fall, is disregarded.\\		 
%		\tabitem Aerodynamic effects acting on the load are neglected.\\
%		\hspace{4mm} reference frame.\\
%		\tabitem Assumption \\
%		\hspace{4mm} Details Assumption 2\\
%		\hline
%	\end{tabular}
%	\caption{Modeling assumptions Quadrotor-Load model}
%	\label{tab:mod.assumptionsQRL}
%\end{table}


\section{Geometric Mechanics}\label{sec:mod.geometric}
In Geometric Mechanics the configuration space of systems is a \textit{group manifold} instead of a Euclidean space. The kinetic and potential energies are expressed in terms of this configuration space and their tangent spaces. It explores the geometric structure of a Lagrangian- or Hamiltonian system through the concepts of vector calculus, linear algebra, differential geometry, and non-linear control theory. Geometric mechanics provides fundamental insights into the nonlinear system mechanics and yields useful tools for dynamics and control theory.

Euler angles are kinematically singular since the transformation from their time rates of change to the angular velocity vector is not globally defined. Furthermore, when angular errors are large, the difference in Euler angles is no longer a good metric to define the orientation error. Local coordinates often require symbolic computational tools due to complexity of multi-body systems. Hence, the error is rather written as the required rotation to get from the current to a desired orientation. As a result, the equations of motion and the control systems can be developed on a configuration manifold in a coordinate-free, compact, unambiguous manner, while singularities of local parameterization are avoided.

To illustrate the difference in configuration spaces, an example is given of a 2-link arm, where the configuration can be expressed by 2 coordinates, see in Figure \ref{fig:mod.armmanifold}. Figure \ref{fig:mod.armcartesian} represents the configuration space as a Cartesian space, where the same dots represent one of the many identical configurations. This shows that this representation suffers from singularities caused by multiple points in one representation being mapped onto a single point in another representation. Figure \ref{fig:mod.armtorus} shows the configuration space as a geometric shape called a \textit{torus}, a manifold where every configuration is mapped uniquely.
\begin{figure}[h!]
	\centering
	%ADD figure Robot Arm
	%ADD figure cartesian (q1,q2), with the red dot representing sam point
	%ADD figure torus (q1,q2)
	\makebox[.3\textwidth][c]{\subfloat[][\label{fig:mod.arm}]{\includegraphics[width=.3\textwidth]{./StyleStuff/dcsc.png}}}
	\makebox[.3\textwidth][c]{\subfloat[][\label{fig:mod.armcartesian}]{\includegraphics[width=.3\textwidth]{./StyleStuff/dcsc.png}}}
	\makebox[.3\textwidth][c]{\subfloat[][\label{fig:mod.armtorus}]{\includegraphics[width=.3\textwidth]{./StyleStuff/dcsc.png}}}
	\caption{Configuration Space of a 2-link arm\label{fig:mod.armmanifold}}
\end{figure}		

%CHECK
%Mechanics studies the dynamics of physical bodies acting under forces and potential fields. 
%In Lagrangian mechanics, the trajectories are obtained by finding the paths that minimize the integral of a Lagrangian over time, called the action integral. 
%Rigid body dynamics are characterized by Lagrangian/Hamiltonian dynamics. The dynamics of a Lagrangian system has unique geometric properties and these are exploited to obtain Euler-Lagrange equations. The resulting intrinsic form of the Euler-Lagrange equations are more compact than equations expressed in terms of local coordinates.


%ADD PROS CONS
%Problems, singularities with Euler-Angles\\
%Other attitude representations, such as exponential coordinates, quaternions, or Euler
%angles, can also be used following standard descriptions, but each of the representations has a disadvantage
%of introducing an ambiguity or singularity.
%Why charts on $ SO(3) $ \url{https://en.wikipedia.org/wiki/Charts_on_SO(3)}\\


\paragraph{Manifolds}
%ADD  manifolds
The fundamental object of differential geometry a manifold. A manifold is a mathematical space, a collection of points, that locally resembles Euclidean space near each point. Examples are a plane, a ball, a torus and a sphere. Manifolds are important objects in mathematics and physics because they allow more complicated structures to be expressed and understood in terms of the relatively well-understood properties of simpler spaces. In Figure \ref{fig:mod.manifold} is illustrated that each point of an n-dimensional manifold has a neighborhood that is homeomorphic to the n-dimensional Euclidean space, meaning that there is a continuous function describing the relation between these spaces.

\begin{figure}[h!]
	%ADD figure of local space on manifold to cartesian space. Sphere onto R2
	%CHECK \url{https://en.wikipedia.org/wiki/Differentiable_manifold}
	\centering
	\makebox[\textwidth][c]{\includegraphics[width=.45\textwidth]{./StyleStuff/dcsc.png}}
	\caption{A manifold locally resembles a Euclidean space\label{fig:mod.manifold}}
\end{figure}

%ADD 
%An Introduction to Differentiable Manifolds is given by \cite{Boothby2003}. 

A differentiable manifold is a smooth and continuous manifold and is locally similar enough to a linear space to allow to do calculus. One can define directions, tangent spaces, and differentiable functions on such a manifold. Each point of an n-dimensional differentiable manifold has a tangent space, which is an n-dimensional Euclidean space consisting of the tangent vectors of the curves that pass through that point. In Figure \ref{fig:mod.tspace} the manifold $ \mathbb{S}^2 $ represents as a sphere, with a tangent space at point $ x $, denoted by $ T_x\mathbb{S}^2 $. Taking the derivative at a point on a manifold is equivalent to a tangent vector at that point. Meaning that derivatives are conceptually equivalent to an infinitesimally short tangent vector. 
\begin{figure}[h!]
	\centering
	%ADD fig sphere with tangent bundle at x
	\makebox[.49\textwidth][c]{\subfloat[][Representation of a manifold with a tangent space \label{fig:mod.tspace}]{\includegraphics[width=.45\textwidth]{./StyleStuff/dcsc.png}}}
	%ADD fig space with SO(3) and so(3)
	\makebox[.49\textwidth][c]{\subfloat[][Identity map of $ SO(3) $ with Lie Algebra $ \mathfrak{so}(3) $ \label{fig:mod.so3}]{\includegraphics[width=.45\textwidth]{./StyleStuff/dcsc.png}}}
	\caption{\label{fig:}}
\end{figure}		

\subparagraph{Configuration Spaces}
Rotation matrices are used to provide a global representation of the attitude of a rigid body, by mapping a representation of vectors expressed in \BF to a representation expressed in \IF \cite{Chaturvedi2011,Murray1994}. 
The configuration of the \a{qr} attitude is a rotation matrix $ R $ in the Special Orthogonal Group $ SO(3) $ defined as
\begin{equation}\label{eq:SO3}
SO(3) \triangleq \left\lbrace R\in\mathbb{R}^{3\times3}|RR^T=I_{3\times3}, det(R)=1\right\rbrace 
\end{equation}

%CHECK nodig?
%The configuration manifold for combined translational and rotational motion of a rigid body is the special Euclidean group $ SE(3) $, which is a semi-direct product of $ SO(3) $ and $ \mathbb{R}^3 $.

$ SO(3) $ is the group of all rotations about the origin of a 3-D Euclidean space, which preserves the origin, Euclidean distance and orientation.
Every rotation has a unique inverse rotation and the identity map satisfies the definition of a rotation. The elements of \textit{Lie Algebra} $ \mathfrak{so}(3) $, a property associated with $ SO(3) $, are the elements of the tangent space of $ SO(3) $ at the identity element, see Figure \ref{fig:mod.so3}. 
These elements define the relation between the rotation $ R $ and its derivative $ \dot{R} $, such that
\begin{equation}\label{eq:Rdot}
\dot{R} = R\hat{\Omega}
\end{equation}
For $ n\in \mathbb{N} $, $ \mathfrak{so}(3) $ is is the vector space of skew-symmetric matrices in $ \mathbb{R}^{n\times n} $ and defined as
\begin{equation}\label{eq:so3}
\mathfrak{so}(n) \triangleq \left\lbrace S\in \mathbb{R}^{n\times n}|S^T=-S\right\rbrace
\end{equation}

The linear map $ \hat{\cdot}:\mathbb{R}^3\rightarrow\mathfrak{so}(3) $ is an isomorphism between $ \mathbb{R}^3 $ and the set of $ 3\times 3 $ skew symmetric matrices. $ \cdot^\vee:\mathfrak{so}(3)\rightarrow\mathbb{R}^3 $ denotes the inverse isomorphism. The mapping between the body angular velocity vector $ \Omega\in\mathbb{R}^3 $ and  $ \hat{\Omega}\in\mathfrak{so}(3) $ can be written as
\begin{equation}\label{eq:mod.hatOmega}
\hat{\Omega}=\begin{bmatrix}
0&-\Omega_3&\Omega_2\\
\Omega_3&0&-\Omega_1\\
-\Omega_2&\Omega_1&0
\end{bmatrix},
\quad
\begin{bmatrix}
0&-\Omega_3&\Omega_2\\
\Omega_3&0&-\Omega_1\\
-\Omega_2&\Omega_1&0
\end{bmatrix}^\vee = \Omega
\end{equation}

The configuration space of the load is represented on a 2-sphere, defined as
\begin{equation}\label{key}
\mathbb{S}^2 \triangleq \left\lbrace q\in\mathbb{R}^{3}|q\cdot q=1\right\rbrace 
\end{equation}
%\begin{equation}\label{key}
%\mathbb{S}^2 \triangleq \left\lbrace (x_1, x_2, x_3)\in\mathbb{R}^{3}| \parallel (x_1, x_2, x_3)\parallel=1\right\rbrace 
%\end{equation}
%ADD sphere - tangent space - perpendicular : angular velocity
\begin{equation}\label{key}
\dot{q} = \omega\times q
\end{equation}
where $ \omega $ is the angular velocity of the suspended load.

\section{Quadrotor-Load Model}	\label{sec:mod.QRLmod}
The Quadrotor-Load model is shown in Figure \ref{fig:QRLmodel}, where the unit vector \lsymb{$ q$}{Unit vector from \a{qr} to Load} gives the direction from the \a{qr} to the Load expressed in \BF. The focus lies on the subsystem where the cable tension is considered to be non-zero. The position of the \a{qr} and Load are related by
\begin{equation}\label{eq:xQ2xL}
x_Q=x_L-Lq
\end{equation}
where \lsymb{$ x_Q $}{Position of the  of the \a{qr} CoM} is the position of the \a{com}, \lsymb{$ x_L $}{Position of the load} is the position of the load, and \lsymb{$ L $}{Length of the cable} is the length of the cable.
%The derived mathematical model is represented by a set of dynamic equations commonly used for rigid body displacements. 

\begin{figure}[h!]
	\centering
	\makebox[\textwidth][c]{\includegraphics[width=.45\textwidth]{./StyleStuff/dcsc.png}}
	\caption{Quadrotor-Load model representation\label{fig:QRLmodel}}
\end{figure}		

Dynamics and optimal control problems for rigid bodies are studied in \cite{Lee2008}, incorporating their geometric features. The focus lies on obtaining geometric properties of the dynamics of rigid bodies, how their configuration can be described and how these geometric properties are utilized in control system analysis and design. 

Considering the properties of the system, the \a{qr} is described as a rigid body with six degrees of freedom, driven by forces and moments. 
%Which means that the motion of a rigid body can be described by a translation of the \acf{com} and a rotation about the \a{com}. 
The configuration of the \a{qr} can be described by the location of its \a{com} and its attitude, which are described in Euclidean space $x_Q\in \mathbb{R}^3 $ and in a nonlinear space $R\in SO(3) $, respectively. 
The configuration of the load can also be described by its location and attitude, described in Euclidean space $x_L\in \mathbb{R}^3 $ and on a two-sphere $ q\in \mathbb{S}^2 $.

The \a{qr} attitude kinematics equation is given by
\begin{equation}\label{key}
\dot{R}=R\hat{\Omega}
\end{equation}
where $ \Omega\in\mathbb{R}^3 $ is the angular velocity represented in the body fixed frame, see Equation \ref{eq:mod.hatOmega}.
%\begin{equation}\label{key}
%\hat{\Omega}=\begin{bmatrix}
%0&-\Omega_3&\Omega_2\\
%\Omega_3&0&-\Omega_1\\
%-\Omega_2&\Omega_1&0
%\end{bmatrix}
%\end{equation}

%LOAD ATTITUDE DYNAMICS
To develop the Euler-Lagrange equations for mechanical systems that evolve on a Lie group, an approach developed by \cite{Lee2008,Lee2005,Lee2009,Lee2011} is used, which is based on Hamilton's principle. 
	
%ADD Hamilton's principle uitleg?

The action integral is defined as
\begin{equation}\label{eq:actionintegral}
S=\int_{t_1}^{t_2}\mathcal{L}dt
\end{equation}
where $\mathcal{L}=\mathcal{T}-\mathcal{U} $ is the Lagrangian of the system, where $\mathcal{T},\mathcal{U}$ are the kinetic and potential energy, respectively. Hamilton's principle of least action states that the path a conservative mechanical system takes between two configurations $ q_1 $ and $ q_2 $ at time $ t_1 $ and $ t_2 $, is the one for which Equation \ref{eq:actionintegral} is an extremum, stated as
\begin{equation}\label{eq:HamPr}
\delta S=\int_{t_1}^{t_2}\delta\mathcal{L}dt=0
\end{equation}
where $ \delta\mathcal{L} $ is the variation of the Lagrangian. For systems with non-conservative forces and moments, Equation \ref{eq:HamPr} is extended to
\begin{equation}\label{eq:HamPrNon}
\delta S=\int_{t_1}^{t_2}(\delta W+\delta\mathcal{L})dt=0
\end{equation}
where $ \delta W $ is the virtual work. Equation \ref{eq:HamPrNon} is applied to the QR-Load system, where the configuration manifold is $ \mathbb{R}^3\times \mathbb{S}^2\times SO(3) $. With the following states
\begin{equation}\label{key}
\textbf{x}= \begin{bmatrix}x_L& \dot{x}_L& q& \omega&R&\Omega
\end{bmatrix}^T
\end{equation}


\paragraph{Variations in energy}

%ADD
Looking at the variations on manifolds\\
This is different from traditional mechanics

%ADD kinetic energy
\begin{equation}\label{key}
\mathcal{T}
\end{equation}

%ADD potential energy
\begin{equation}\label{key}
\mathcal{U}
\end{equation}

%ADD
%We approximate the variation in kinetic energy using the first-order Taylor approximation
\begin{equation}\label{key}
\delta\mathcal{T}
\end{equation}

%ADD
%We approximate the variation in potential energy using the first-order Taylor approximation
\begin{equation}\label{key}
\delta\mathcal{U}
\end{equation}

Terms of virtual work; thrust working on \a{qr}'s \a{com}
\begin{equation}\label{key}
\delta W_1
\end{equation}

The quadrotor is the only rigid body in the subsystem and is acted on by moment M, with associated virtual work:
\begin{equation}\label{key}
\delta W_2
\end{equation}

Eqs. 21 - 24 allow us to rearrange Eq. 14 as variations in
each generalized coordinate:
\begin{equation}\label{key}
\delta S
\end{equation}

$ \delta R $ and $ \delta q $ vary on manifolds. Variations can be described as
\begin{equation}\label{key}
\begin{aligned}
\delta R&=\\
\delta q&=
\end{aligned}
\end{equation}

From Equation \ref{eq:CF} the following variations can be obtained
\begin{equation}\label{key}
\begin{aligned}
\delta \dot{q}&=\\
\delta \dot{R}&=\\
\delta \hat{\Omega}&=\\
&=\\
&=\\
\delta\Omega&=
\end{aligned}
\end{equation}


%CHECK nodig?
%***************************************\\
%Rigid Body Attitude Dynamics evolve on $ SE(3) $.
%\begin{align}\label{eq:eomrigidbody}
%%CHECK waar komt deze equation vandaan?
%J\dot{\Omega}+\Omega\times J\Omega &= mg\rho\times R^Te_3+u\\ 
%\dot{R} &= R\hat{\Omega}
%\end{align}
%***************************************\\

%CHECK standard newton euler
%***************************************\\
%The equations of motion for a rigid body with configuration $ SE(3) $ are given by the \textit{Newton-Euler equations} \cite{Murray1994}:
%\begin{equation}\label{key}
%\begin{bmatrix}
%	mI&0\\
%	0&\mathcal{I}
%\end{bmatrix}
%\begin{bmatrix}
%	\dot{v}^b\\
%	\dot{\omega}^b
%\end{bmatrix}+
%\begin{bmatrix}
%	\omega^b\times mv^b\\
%	\omega^b\times\mathcal{I}\omega^b
%\end{bmatrix}=F^b
%\end{equation}
%where $ m $ is the mass of the body, $ \mathcal{I} $ is the inertia tensor, and $ V^b=(v^b,\omega^b) $ and $ F^b $ represent the instantaneous body velocity and applied body wrench.
%***************************************\\


\section{Classical Modeling}\label{sec:mod.clas}

%CHECK waarom dit? Per se nodig om LQR uit te leggen?
This section describes the derivation of the model by using classical modeling techniques.

***************************************\\
%CHECK is deze reference nog relevant?
Reference \ref{app:model}

***************************************\\

When assuming small angle maneuvers, \textit{Euler-angles} can be used to locally parameterize the orientation of the body-fixed reference coordinate frame with respect to the inertial reference coordinate frame. Simple linear controllers are often based on a linearized dynamical model, applying this small angles assumption. 

From Newton's law follows
\begin{equation}\label{eq:newton}
\begin{aligned}
\dot{x}_Q &= v_Q\\
m_Q\dot{v}_Q &=fRe_3-m_Qge_3-Tq\\
\dot{x}_L &= v_L\\
m_L\dot{v}_L &=-m_Lge_3+Tq
\end{aligned}
\end{equation}
%which gives the following equation, derived in Section \ref{sec.app:loaddyn},
%\begin{equation}\label{key}
%%CHECK whether equation is correct
%(m_Q+m_L)(\dot{v}_L+ge_3)=fRe_3-m_QL\ddot{q}
%\end{equation}


Because the Euler-Angles are used, a function is required that maps a vector of the Z-X-Y Euler angles to its rotation matrix $ R\in SO(3) $, which is denoted as \cite{Mahony2012}
\begin{equation}\label{key}
R_{312}({\phi},{\theta},{\psi})=\begin{bmatrix}
c_{\psi}c_{\theta}-s_{\phi}s_{\psi}s_{\theta}&-c_{\phi}s_{\psi}&c_{\psi}s_{\theta}+c_{\theta}s_{\phi}s_{\psi}\\
c_{\theta}s_{\psi}+c_{\psi}s_{\phi}s_{\theta}&c_{\phi}c_{\psi}&s_{\psi}s_{\theta}-c_{\psi}c_{\theta}s_{\phi}\\
-c_{\phi}s_{\theta}&s_{\phi}&c_{\phi}c_{\theta}
\end{bmatrix}
\end{equation}
The Z-X-Y Euler angles to model the rotation can be seen in Figure \ref{fig:mod.modelQRtrad}. The first rotation by yaw angle $ \psi $ is around the z-axis of \IF. Next is the rotation by roll angle $ \phi $, and the last rotation is by pitch angle $ \theta $.
\begin{figure}[h!]
	\centering
	\makebox[\textwidth][c]{\includegraphics[width=.45\textwidth]{./StyleStuff/qrmodelppt.png}}
	\caption{Quadrotor-Load model representation\label{fig:mod.modelQRtrad}}
\end{figure}

The unit vector $ q $ from the \a{qr} to the load is represented in \BF. Define $ \phi_L $ as the yaw-rotation of the load around the z-axis of \BF and $ \theta_L $ as the angle between the cable and the z-axis of \BF, see Figure \ref{fig:mod.modelQRLtrad}.
\begin{figure}[h!]
	%ADD figure with angles phiL and theta L
	\centering
	\makebox[\textwidth][c]{\includegraphics[width=.45\textwidth]{./StyleStuff/dcsc.png}}
	\caption{\label{fig:mod.modelQRLtrad}}
\end{figure}	

\begin{equation}\label{eq:q}
%		q=-R_{\psi_L}R_{\theta_L}e_3=\begin{bmatrix}
%		s_{\theta_L}c_{\psi_L}\\s_{\theta_L}s_{\psi_L}\\-c_{\theta_L}
%		\end{bmatrix}
%\begin{align}
q=\begin{bmatrix}
s_{\theta_L}c_{\phi_L}\\
s_{\theta_L}s_{\phi_L}\\
c_{\theta_L}
\end{bmatrix}
%\dot{q}&=\begin{bmatrix}
%s_{\theta_L}c_{\psi_L}\\
%s_{\theta_L}s_{\psi_L}\\
%c_{\theta_L}
%\end{bmatrix}\\
%\end{align}
\end{equation}  
Differentiating Equation (\ref{eq:xQ2xL}) and (\ref{eq:q}) gives
\begin{equation}\label{key}
\begin{aligned}
\ddot{x}_L&=\ddot{x}_Q-\ddot{q}L\\
\ddot{q}&=\begin{bmatrix}
\ddot{\theta}_Lc_{\theta_L}c_{\phi_L}-\ddot{\phi}_Ls_{\theta_L}s_{\phi_L}-\dot{\phi}_L^2s_{\theta_L}c_{\phi_L}-\dot{\theta}_L^2s_{\theta_L}c_{\phi_L}-2\dot{\theta}_L\dot{\phi}_Lc_{\theta_L}s_{\phi_L}\\
\ddot{\theta}_Lc_{\theta_L}s_{\phi_L}+\ddot{\phi}_Ls_{\theta_L}c_{\phi_L}-\dot{\phi}_L^2s_{\theta_L}s_{\phi_L}-\dot{\theta}_L^2s_{\theta_L}s_{\phi_L}+2\dot{\theta}_L\dot{\phi}_Lc_{\theta_L}c_{\phi_L}\\
-\ddot{\theta}_Ls_{\theta_L}-\dot{\theta}_L^2 c_{\theta_L}\\
\end{bmatrix}
\end{aligned}
\end{equation}

\begin{equation}\label{key}
\begin{aligned}
\ddot{x}_Q&=\frac{1}{m_Q}(f(c_{\psi}s_{\theta}+c_{\theta}s_{\phi}s_{\psi})-Ts_{\theta_L}c_{\psi_L})\\
\ddot{y}_Q&=\frac{1}{m_Q}(f(s_{\psi}s_{\theta}-c_{\psi}c_{\theta}s_{\phi})-Ts_{\theta_L}s_{\psi_L})\\
\ddot{z}_Q&=\frac{1}{m_Q}(f(c_{\phi}c_{\theta})-Tc_{\theta_L})-g\\
\end{aligned}
\end{equation}

\begin{align}\label{key}
%CHECK wat is hier de bedoeling van? Checken in Garcia of literatuur?
	\ddot{\psi}&=\tilde{\tau}_{\psi}\\
\ddot{\theta}&=\tilde{\tau}_{\theta}\\
\ddot{\phi} &=\tilde{\tau}_{\phi}
\end{align}

\section{Stability Analysis}
Lyapunov Analysis on SO3 x R3 and S2 x R3
Closed-loop full-attitude dynamics evolve on the non- Euclidean manifold SO3 x R3, while closed-loop re-
duced-attitude dynamics evolve on the non-Euclidean mani- fold S2 x R3. Since these manifolds are locally Euclidean, local stability properties of a closed-loop equilibrium solution can be assessed using standard Lyapunov methods. In ad- dition, the LaSalle invariance result and related Lyapunov results apply to closed-loop vector fields defined on these
manifolds. However, since the manifolds SO132 and S2 are compact, the radial unboundedness assumption cannot be satisfied; consequently, global asymptotic stability cannot fol- low from a Lyapunov analysis on Euclidean spaces [40], and therefore must be analyzed in alternative ways [19]–[23].\cite[p.43]{Chaturvedi2011}

%HOeft waarschijnlijk niet eens een section te zijn, kan kort en bondig
\section{Summary}

***************************************\\
Compact, unambiguous, globally defined, 

Pro/Cons of Classical Modeling Techniques vs Geometric Modeling\\

Linearized model/State Space model vs. Geometric modeling\\


Geometric Mechanics/Lie Groups/Lie Algebra is used in order to represent the dynamics of the system onto the nonlinear configuration manifold $ SE(3) $\\
Advantage of this method is\\
Enables to model on \\
That type of control is discussed in the next chapter


***************************************\\
\chapter{Control Design} \label{ch:control}
%ADD intro: in this section etc
Section \ref{sec:con.nlgc} introduces the concepts of Nonlinear Geometric Control, and the control design structure is discussed. 
The required control inputs are calculated by defining the tracking errors on nonlinear manifolds similar to the configuration spaces of the system dynamics.

%For the control of the different flight modes, the controllers are designed on the nonlinear geometric space, 
%being functions of the earlier described tracking errors. 
%Different flight modes are necessary to control the under-actuated system. 
A backstepping control approach is applied for the control of the load position tracking problem, allowing different controllers to operate in a cascaded structure. The controllers that are designed for each flight mode, are discussed in Section \ref{sec:con.track}. 

%ADD 
%deze references introduceren
%\cite{Bullo2005,Jurdjevic1997}

***************************************\\
%BART
bart: Descibe just as in chapter 2 what we are going to read in the next sections\\
nam: please check

***************************************\\

\section{Nonlinear Geometric Control}\label{sec:con.nlgc}
Many control systems are developed for the standard form of ordinary differential equations, namely $ \dot{x}=f(x,u) $, where the state and the control input are denoted by $ x $ and $ u $. It is assumed that the state and the control input lie in Euclidean spaces, and the system equations are defined in terms of smooth functions between Euclidean spaces. However, for many mechanical systems, the configuration space can only be expressed locally as a Euclidean space. In order to express the configuration space globally, a nonlinear space is required. A dynamical model on this nonlinear space is obtained in the previous chapter.

Geometric Control Theory is the study on how geometry of the state space influences controls problems. 
In control systems engineering, the underlying geometric features of a dynamic system are often not considered carefully. 
Differential geometric control techniques utilize these geometric properties for control system design and analysis.
The objective is to express both the system dynamics and control inputs on manifolds instead of local charts. 
In contrast to locally defined linear control, nonlinear geometric control can be defined almost globally, avoiding singularities that occur in the representation of large angles and complex maneuvering.

Global nonlinear dynamics of various classes of closed loop attitude control systems have been studied in recent years \cite{Chaturvedi2011a}.
In contrast to hybrid control systems \cite{Gillula2010}, complicated reachability set analysis is not required to guarantee safe switching between different flight modes, as the region of attraction for each flight mode covers the configuration space almost globally.


\paragraph{Backstepping Control}\label{sec:con.back}
A backstepping approach, or cascade control, is a Lyapunov based technique to design the control of nonlinear dynamical systems and ensures Lyapunov stability. 
This approach is commonly used for the control of \a{qr}s \cite{Mahony2012} and will also be used in this research for the control of the load trajectory tracking problem.\\
The basic principle is to create a cascaded structure by starting with a stable system as a base, then "stepping back" from this base to add a control loop around it that stabilizes the new system and enables the control of another state. 
This is repeated until the final external control is reached, see Figure \ref{fig:con.loop}.
The control law is designed by using states as virtual control signals. 
Each loop computes a virtual command signal for the adjacent inner loop. 
This backstepping approach determines how to stabilize the \a{qr} with the control inputs $ f $ and $ M $. 
The inner controller determines what the required control inputs are, driven by $ R_c $.  
The next controller calculates how to drive $ R_c $ based on $ q_c $, such that the \a{qr} is stabilized.
And the last controller determines which $ q_c $ is needed to follow the desired load position $ x_{L,d} $.

% CHECK nodig?
%\url{http://www.control.lth.se/media/Education/EngineeringProgram/FRTN05/2013/lec09_2013eight.pdf}
%We want to design a state feedback u=u(x) that stabilizes
%\begin{equation}\label{key}
%\begin{aligned}
%\dot{x}_1&=f(x_1)+g(x_1)x_2\\
%\dot{x}_2&=u
%\end{aligned}
%\end{equation}
%Idea is to see system as a cascade connection. Design controller first for inner loop, then for the outer.
%
%Back-Stepping Lemma
%\textbf{Lemma}: Let $ z=\begin{pmatrix}x_1,\cdots,x_{k-1}\end{pmatrix}^T$ and 
%\begin{equation}\label{key}
%\begin{aligned}
%\dot{z}&=f(z)+g(z)x_k\\
%\dot{x}_k&=u
%\end{aligned}
%\end{equation}
%Assume $ \phi(0)=0 $, $ f(0)=0 $,
%\begin{equation}\label{key}
%\dot{z}=f(z)+g(z)\phi(z)
%\end{equation}
%stable, and $ V(z) $ a Lyapunov function (with $ \dot{V}\leq-W $). Then, 
%\begin{equation}\label{key}
%u=\frac{d\phi}{dz}\begin{pmatrix}
%f(z)+g(z)x_k
%\end{pmatrix}-\frac{dV}{dz}g(z)-(x_k-\phi(z))
%\end{equation}
%stabilizes $ x=0 $ with $ V(z)+(x_k-\phi(z))^2/2 $ begin a Lyapunov function.
%
%The backstepping approach determines how to stabilize the {\displaystyle \mathbf {x} } \mathbf {x}  subsystem using {\displaystyle z_{1}} z_{1}, and then proceeds with determining how to make the next state {\displaystyle z_{2}} z_{2} drive {\displaystyle z_{1}} z_{1} to the control required to stabilize {\displaystyle \mathbf {x} } \mathbf {x} . Hence, the process "steps backward" from {\displaystyle \mathbf {x} } \mathbf {x}  out of the strict-feedback form system until the ultimate control {\displaystyle u} u is designed.
%***************************************\\

Because the \a{qr} has only four actuators, it is not possible to control all \a{DOF}s of the \a{qr}-Load system simultaneously. The backstepping approach allows control of different flight modes in which parts of the \a{DOF}s are controlled. The flight modes and their functions are defined in order, from the most inner loop to the most outer loop, as follows
\begin{outline}
\1 QR Attitude Controlled Mode 
\2 Track a desired QR attitude $ R_d(t) $ and a heading direction $ b_{1_d}(t) $
\2 Give a desired input $ M $ for system
\1 Load Attitude Controlled Mode 
\2 Track a desired load attitude command $ q_d(t) $
\2 Give a computed \a{qr} attitude $ R_c $ for the \a{qr} attitude controller (instead of $ R_d(t) $)
\1 Load Position Controlled Mode
\2 Track a desired load position $ x_{L,d}(t) $
\2 Give a computed load attitude $ q_c $ for the load attitude controller (instead of $q_d(t) $)
\end{outline}
where the subscript $d $ denotes a desired tracking reference, and the subscript $ c $ denotes a computed value that is calculated as a tracking reference. The difference in this notation is whether the signal is a predefined desired signal, or a signal computed by a controller.

%The controller that is used in this research is shown in Figure \ref{fig:con.loop}. The lowest levels have the highest bandwidth and are in control of the rotor rotational speeds $ \omega_i $, the total force $ f $ and moments $ M $. The next level controls the load attitude $ q $, and the top level controls the load position $ x_L $. 
\begin{figure}[h!]
	\centering
	\makebox[\textwidth][c]{\includegraphics[trim={0 0 0 9cm},clip,width=.95\textwidth]{./StyleStuff/backstepQR2.png}}
	\caption{Nonlinear Geometric Control Loop of the QR-Load system \cite{Sreenath2013c}\label{fig:con.loop}}
\end{figure}	

The design of the controllers for the \a{qr} attitude can be found in \cite{Lee2010}, and the controllers of load attitude- and position this can be found in \cite{Sreenath2013c}. Thorough stability analyses are presented in either references. For a deeper understanding of Lyapunov stability analysis in geometric control, the reader can refer to \cite{Bullo2005}.

\paragraph{Configuration Errors}\label{sec:con.errors}
The control system for a trajectory tracking problem requires state feedback and a measure of errors defined by a difference between the desired and current states.
The system dynamics evolve on nonlinear manifolds that describe the configuration spaces for the \a{qr} attitude $ \in SO(3) $ and the load attitude $ \in \mathbb{S}^2 $. 
Tracking errors functions are defined on these same manifolds, as shown in \cite{Bullo2005}, and form the basis of the controllers that must stabilize the \a{qr}-Load system.
%Likewise, the tracking errors $ \Psi_R:SO(3)\times SO(3)\rightarrow\mathbb{R} $ and $ \Psi_q:\mathbb{S}^2\times \mathbb{S}^2\rightarrow\mathbb{R} $ are expressed on these manifolds. 
%an for the purpose of control design. 
%The derivation of the error functions can be found in \cite{Bullo2005}.
 
%CHECK 
%\cite{Maithripala2006}

%CHECK
%Attitude control systems naturally evolve on nonlinear configuration spaces such as $ \mathbb{S}^2 $ and $ SO(3) $. 
%Attitude tracking control is developed on $ SO(3) $, therefore it avoids singularities of Euler-Angles.
Recall that $ R $ is the rotation matrix to describe the \a{qr} attitude, and $ R_d $ is the desired rotation matrix. To describe the relative rotation from the body frame to the desired frame, the attitude error is denoted as $ R^T_dR $. 
The \a{qr} attitude error function $ \Psi_R $ on $ SO(3) $ is chosen to be \cite{Lee2010c}
\begin{equation}\label{eq:psiR}
\Psi_R(R,R_d)=\frac{1}{2}tr\left[I-R_d^TR\right]
\end{equation}
such that $ \Psi_R $ is locally positive-definite about $ R^T_dR=I $ within the region where the rotation angle between $ R $ and $ R_d $ is less than $ 180^\circ $. 
It can be shown that this region where $ \Psi_R<2 $ almost covers $ SO(3) $.
%ADD explain
% instead of comparing all elements of rotation matrix. PsiR is a measure for the error
%ADD 
%(physical) Meaning of the error functions
Equation \ref{eq:mod.varRq} states that the variation of the rotation matrix is expressed as $ \delta R = R\hat{\eta} $ for $ \eta\in\mathbb{R}^3 $. Such that with Equation \ref{eq:mod.hatvee}, the derivative of Equation \ref{eq:psiR} is given by 
\begin{equation}\label{key}
\begin{aligned}
\mathbf{D}_R\Psi(R,R_d)\cdot R\hat{\eta}&=-\frac{1}{2}tr[R_d^TR\hat{\eta}]\\
&=\frac{1}{2}(R^T_dR-R^TR_d)^\vee\cdot\eta
\end{aligned}
\end{equation}
%By applying Equation \ref
%\texttt{\begin{equation}\label{key}
%\mathbf{D}_R\Psi(R,R_d)\cdot R\hat{\eta}=
%\end{equation}}
where the \textit{vee map} $ ^\vee:\mathfrak{so}(3)\rightarrow\mathbb{R}^3 $ is the inverse of the \textit{hat map} defined in Section \ref{sec:mod.geometric}. 
From this derivative, the attitude tracking error $ e_R $ is obtained 
\begin{equation}\label{eq:con.eR}
e_R=\frac{1}{2}(R_d^TR-R^TR_d)^\vee
\end{equation}
%The tracking error functions on $ TSO(3) $, the tangent space of $ SO(3) $, are defined as
%The attitude and angular velocity tracking error should be carefully chosen as they evolve on the tangent bundle of  $ SO(3) $. \cite{Lee2010c} 
It is important to note that the tangent vectors $ \dot{R} $ and $ \dot{R}_d $ cannot be compared directly, since they do not lie in the same space. $ \dot{R} $ and $ \dot{R}_d $ are expressed in their own tangent spaces, denoted by $ T_RSO(3)$ and $ T_{R_d}SO(3)$, respectively. For this reason, $ \dot{R}_d $ is transformed into a vector on $ T_RSO(3) $ to compare it with $ \dot{R} $. This is done by an mathematical object called a \textit{transport map}, which enables the comparison of velocities living in different spaces. See Figure \ref{fig:con.transport}, where two curves $ R(t), R_d(t)$ evolve on manifold $ SO(3) $, such that transport map $ \mathcal{T}(R,R_d):T_{R_d}SO(3)\mapsto T_RSO(3) $ allows comparison of $ \dot{R} $ and $ \dot{R}_d $.
\begin{figure}[h!]
	\centering
	\makebox[\textwidth][c]{\includegraphics[width=.45\textwidth]{./StyleStuff/transport.png}}
	\caption{Transport map $ \mathcal{T}(R,R_d) $\label{fig:con.transport}}
\end{figure}		

%BUllo p536 transport maps
%BULLO Bullo p555 tracking error
%Substituting Equations \ref{eq:SO3} and \ref{eq:Rdot} into 
The comparison of the two tangent vectors is needed to calculate the error of body angular velocity $ \Omega $. This is derived from the velocity error that corresponds to the transport map $ \mathcal{T}(R,Rd)$, which is defined as
\begin{equation}\label{key}
\dot{e}=\dot{R}-\dot{R}_d(R_d^TR)
\end{equation}
This can be rewritten as follows
\begin{equation}\label{key}
\begin{aligned}
\dot{R}-\dot{R}_d(R_d^TR) &=R\hat{\Omega}-R_d\hat{\Omega}_d(R_d^TR) \\
&=R(\Omega)^\wedge-(RR^T)R_d\hat{\Omega}_dR_d^TR\\
&=R(\Omega)^\wedge-R(R^TR_d{\Omega}_d)^\wedge \\
&=R(\Omega-R^TR_d{\Omega}_d)^\wedge 
\end{aligned}
\end{equation}
From this follows the angular velocity tracking error $ e_{\Omega} $ in \BF, which  is defined as
\begin{align}\label{eq:con.eOmega}
e_\Omega&=\Omega- R^TR_d\Omega_d
\end{align}
Similar to the form of Equation \ref{eq:mod.R}, $ e_\Omega $ is the angular velocity vector of the relative rotation matrix $ R_d^TR $, represented in \BF. 
It can be shown that the following equation holds
\begin{equation}\label{key}
\frac{d}{dt}(R^T_dR)=(R_d^TR)\hat{e}_\Omega
\end{equation}

Next, the load attitude error function is expressed on $ \mathbb{S}^2 $ and represents the distance from the direction $ q $ to the desired direction $ q_d $. 
This is given by 
\begin{equation}\label{eq:psiq}
\Psi_q=1-q_d^Tq
\end{equation}

In the same fashion a \textit{transport map} is used for a comparison between the tangent vectors on tangent spaces $ T_q\mathbb{S}^2$ and $ T_{q_d}\mathbb{S}^2$. This results in the following error functions on $ T\mathbb{S}^2 $
\begin{align}
e_q&=\hat{q}^2q_d\label{eq:con.eq}\\
e_{\dot{q}}&=\dot{q}-(q_d\times\dot{q}_d)\times q\label{eq:con.edq}
\end{align}

The tracking errors for position and velocity are defined as
\begin{align}\label{key}
e_x&=x-x_d\\
e_v&=v-v_d
\end{align}
where $ v_d=\dot{x}_d $ and $ x_d(t) \in \mathbb{R}^3$ is a smooth desired load position.

\section{Tracking modes}\label{sec:con.track}
\subsection{Quadrotor Attitude Tracking}\label{sec:con.qratt}
The QR Attitude Controlled Mode is designed to control the \a{qr} attitude by tracking a smooth desired \a{qr} attitude command $ R_d(t) $.
This is done by controlling the earlier obtained error dynamics of $ e_R $ and $ e_\Omega $. The derivations of the equations in this section can be found in Section \ref{sec:app.error}.\\
From Equations \ref{eq:con.eR} and \ref{eq:con.eOmega}, the derivative of the attitude tracking error $ e_R $ can be written as
\begin{equation}\label{key}
\dot{e}_R=\frac{1}{2}(R_d^TR\hat{e}_\Omega+\hat{e}_\Omega R^TR_d)^\vee
\end{equation}

Similar to Equation \ref{eq:mod.R}, the kinematics equation for the desired attitude can be written as
\begin{equation}\label{eq:con.dotRd}
\dot{R}_d=R_d\hat{\Omega}_d \text{, and so } \hat{\Omega}_d=R_d^T\dot{R}_d
\end{equation}
The definition of the desired angular acceleration $ \dot{\Omega}_d $ can then be defined as follows
\begin{equation}\label{key}
\begin{aligned}
\dot{\hat{\Omega}}_d&=(\dot{R}_d^T\dot{R}_d)+(R_d^T\ddot{R}_d)\\
&=(R_d\hat{\Omega}_d)^T(R_d\hat{\Omega}_d)+(R_d^T\ddot{R}_d)\\
&=-\hat{\Omega}_d\hat{\Omega}_d+R_d^T\ddot{R}_d,\\
\dot{\Omega}_d&=(-\hat{\Omega}_d\hat{\Omega}_d+R_d^T\ddot{R}_d)^\vee
\end{aligned}
\end{equation}
From previous equations and Equations \ref{eq:mod.R}, \ref{eq:con.eOmega} and the fact that $ \hat{\Omega}_d\Omega_d =0$, follows that the derivative of the angular velocity tracking error $ e_\Omega $ can be written as 
\begin{equation}\label{eq:con.deOmega}
\dot{e}_\Omega=\dot{\Omega}+\hat{\Omega}R^TR_d\Omega_d-R^TR_d\dot{\Omega}_d
%\dot{e}_\Omega=J^{-1}(-\Omega\times J\Omega + M)+\hat{\Omega}R^TR_d\Omega_d-R^TR_d\dot{\Omega}_d
\end{equation}
By substituting Equation \ref{eq:mod.qratt} follows
\begin{equation}\label{eq:con.dOmega}
\dot{e}_\Omega=J^{-1}(-\Omega\times J\Omega + M)+\hat{\Omega}R^TR_d\Omega_d-R^TR_d\dot{\Omega}_d
\end{equation}

Now the control input $ M $ can be defined as a proportional term, a derivative term and a canceling term, as follows \cite{Sreenath2013c}
\begin{equation}\label{eq:con.M}
M = -k_Re_R-k_\Omega e_\Omega+\Omega\times J\Omega-J(\hat{\Omega}R^TR_d\Omega_d-R^TR_d\dot{\Omega}_d)
\end{equation}
such that Equation \ref{eq:con.deOmega} is reduced to
\begin{equation}\label{eq:con.JdeOmega}
J\dot{e}_\Omega=-k_Re_R-k_\Omega e_\Omega
\end{equation}
for any positive constants $ k_R, k_\Omega $.
In \cite{Sreenath2013b}, Equation \ref{eq:con.M} is defined as
\begin{equation}\label{eq:con.Meps}
M = -\frac{1}{\epsilon^2}k_Re_R-\frac{1}{\epsilon}k_\Omega e_\Omega+\Omega\times J\Omega-J(\hat{\Omega}R^TR_d\Omega_d-R^TR_d\dot{\Omega}_d)
\end{equation}
where \lsymb{$ \epsilon $}{Tuning parameter to enable rapid exponential convergence of $ e_R, e_\Omega $} is a parameter to enable rapid exponential convergence of the attitude- and angular velocity error functions, such that $ 0<\epsilon<1 $.\\
%%ADD waarom?
%%CHECK waar moet dit
%It can be shown that $ \parallel\dot{e}_R\parallel\leq\parallel e_\Omega \parallel  $ for all $ R^T_dR\in SO(3) $.
 
A stability analysis of the controller is presented in \cite{Lee2010} and it is proven that the zero equilibrium of the closed loop tracking error $ (e_R,e_\Omega)=(0,0) $ is exponentially stable, if the initial conditions satisfy
\begin{equation}\label{eq:dom1}
\Psi_R(R(0),R_d(0))<2
\end{equation}
\begin{equation}\label{eq:dom2}
\parallel e_\Omega(0)\parallel^2<\frac{2}{\lambda_M(J)}\frac{k_R}{\epsilon^2}(2-\Psi_R(R(0),R_d(0)))
\end{equation}
where \lsymb{$ \lambda_M(\cdot) $}{Maximum eigenvalue} denotes the maximum eigenvalue. The domain of attraction is defined by Equations \ref{eq:dom1} and \ref{eq:dom2}. \\
Furthermore, there exist constants $ \alpha_R,\beta_R>0 $ such that
\begin{equation}\label{eq:con.PsiRconv}
\Psi_R(R(t),R_d(t)) \leq min\left\lbrace 2,\alpha_Re^{-\beta_Rt}\right\rbrace 
\end{equation}
%ADD explain that error function will be asymptoticallly stable for right parameters. larger region of attraction

%CHECK what this is about
%Asymptotic tracking of the quadrotor attitude does not require specification of the thrust magnitude. As an auxiliary problem, the thrust magnitude can be chosen in many different ways to achieve an additional translational motion objective. For example, it can be used to asymptotically track a quadrotor altitude command [28]. Since the translational motion of the quadrotor UAV can only be partially controlled; this flight mode is most suitable for short time periods where an attitude maneuver is to be completed. \cite{Goodarzi2015b}

%CHECK waar moet dit?
\begin{equation}\label{key}
J\dot{e}_\Omega=-\frac{1}{\epsilon^2}k_Re_R-\frac{1}{\epsilon}k_\Omega e_\Omega
\end{equation} 

\subsection{Load Attitude Tracking}\label{sec:con.loadatt}

%ADD How is the controller built.
%CHECK nodig? Net als bij M, hoe komen ze hier op?
%A control input is composed of a proportional term along the gradient of $ \Psi_q $, a derivative term and a cancellation term.
%\begin{equation}\label{key}
%u=mL^2(-k_qq_d\times q-k_{\omega}\omega-\frac{g}{L}q\times e_3)
%\end{equation}


%ADD Dependent of what values? 	How to choose parameters.

%\begin{figure}[h!]
%	\centering
%	\makebox[\textwidth][c]{\includegraphics[width=.45\textwidth]{./StyleStuff/dcsc.png}}
%	\caption{\label{fig:con.loadattloop}}
%\end{figure}		

The Load Attitude Controlled Mode tracks a desired load attitude $ q_d $ by calculating a command signal for the \a{qr} attitude, defined as
\begin{equation}\label{eq:con.R}
R_c = \begin{bmatrix}
b_{1c}; b_{3c}\times b_{1c};b_{3c}
\end{bmatrix}
\end{equation}
where $ b_{3c} \in \mathbb{S}^2 $ is defined by 
\begin{equation}\label{eq:con.b3c}
b_{3c}=\frac{F}{||F||}
\end{equation}
Such that $ F $ in Equation \ref{eq:con.b3c} is defined by a normal component $ F_n $, $ F_{pd} $ and $ F_{ff}$
\begin{equation}\label{key}
F=F_n-F_{pd}-F_{ff}
\end{equation}
 Control forces for a system evolving on $ \mathbb{S}^2 $, are derived in \cite{Bullo2005}. 
 This results in a proportional-derivative force $ F_{pd} $ and a feed forward force $ F_{ff} $, that are functions of Equations \ref{eq:con.eq} and \ref{eq:con.edq}. The following terms are obtained
\begin{equation}\label{key}
\begin{aligned}
F_{pd}&=-k_P\hat{q}^2q_d-k_D(\dot{q}-(q_d\times\dot{q}_d\times q)\\
&=-k_qe_q-k_\omega e_{\dot{q}}
\end{aligned}
\end{equation}
\begin{equation}\label{key}
F_{ff}=m_QL\langle\langle q,q_d\times\dot{q}_d\rangle\rangle_{\mathbb{R}^3}(q\times \dot{q})+m_QL(q_d\times \ddot{q}_d)\times q
\end{equation}
The unit vector $ b_{1c} $ is defined as \cite{Lee2010d}
\begin{equation}\label{key}
b_{1c}=-\frac{1}{||b_{3c}\times b_{1d}||}(b_{3c}\times(b_{3c}\times b_{1d}))
\end{equation}
where $ b_{1d}\in \mathbb{S}^2 $ is chosen, not parallel to $ b_{3c} $.
The total upward thrust is defined as
\begin{equation}\label{key}
f=F\cdot Re_3
\end{equation}

%ADD explain that error function will be asymptoticallly stable for right parameters. larger region of attraction
It is proven in \cite{Sreenath2013c} that the zero equilibrium of the closed loop tracking error $ (e_q,e_{\dot{q}},e_R,e_\Omega)=(0,0,0,0) $ is exponentially stable, if the initial conditions satisfy
\begin{equation}\label{eq:dom3}
\Psi_q(q(0),q_d(0))<2
\end{equation}
\begin{equation}\label{eq:dom4}
\parallel e_{\dot{q}}(0)\parallel^2<\frac{2}{m_QL}{k_R}(2-\Psi_q(q(0),q_d(0)))
\end{equation}

The domain of attraction is defined by Equations \ref{eq:dom1}, \ref{eq:dom2}, \ref{eq:dom3} and \ref{eq:dom4}.
Furthermore, there exist constants $ \alpha_q,\beta_q>0 $ such that
\begin{equation}\label{eq:con.Psiqconv}
\Psi_q(q(t),q_d(t)) \leq min\left\lbrace 2,\alpha_qe^{-\beta_qt}\right\rbrace 
\end{equation}

\subsection{Load Position Tracking}\label{sec:con.loadpos}

%CHECK nodig?
%Explain how $ f $ and $ \vec{b_{1_d}} $ is obtained from $ x_d(t) $?\\

Tracks load position reference. Outputs load attitude reference.

\begin{equation}\label{eq:con.q}
q_c = - \frac{A}{||A||}
\end{equation}
where
\begin{equation}\label{key}
A = -k_xe_x-k_ve_v+(m_Q+m_L)(\ddot{x}_{L,d}+ge_3)+m_QL(\dot{q}\cdot\dot{q})q
\end{equation}
with $ e_x=x_L-x_{L,d} $ and $ e_v=\dot{x}_L-\dot{x}_{L,d} $.
Furthermore, $ F_n $ is redefined as
\begin{equation}\label{key}
F_n=(A\cdot q)q
\end{equation}
%\begin{figure}[h!]
%	\centering
%	\makebox[\textwidth][c]{\includegraphics[width=.45\textwidth]{./StyleStuff/dcsc.png}}
%	\caption{\label{fig:con.loadposloop}}
%\end{figure}		

%CHECK error function will be asymptoticallly stable for right parameters? 
It is proven in \cite{Sreenath2013c} that the zero equilibrium of the closed loop tracking error $ (e_x,e_v,e_q,e_{\dot{q}},e_R,e_\Omega)=(0,0,0,0,0,0) $ is exponentially stable, if the initial conditions satisfy
\begin{equation}\label{eq:dom5}
\Psi_q(q(0),q_c(0))<\psi_1<1
\end{equation}
\begin{equation}
\parallel e_{x}(0)\parallel^2<e_{x_{max}}
\end{equation}
where $ e_{x_{max}} $ and $ \psi_1 $ are fixed design depended constants. 

The domain of attraction is defined by Equations \ref{eq:dom1}, \ref{eq:dom2}, \ref{eq:dom5} and the following equation
\begin{equation}
\parallel e_{\dot{q}}(0)\parallel^2<\frac{2}{m_QL}{k_q}(\psi_1-\Psi_q(q(0),q_d(0)))
\end{equation}


Furthermore, there exist constants $ \alpha_q,\beta_q>0 $ such that
\begin{equation}\label{key}
\Psi_q(q(t),q_d(t)) \leq min\left\lbrace 2,\alpha_qe^{-\beta_qt}\right\rbrace 
\end{equation}

\section{Stability Analysis}\label{sec:con.sta}

Normally Lyapunov Analysis is 


Lyapunov Analysis on SO3 x R3 and S2 x R3

Closed-loop full-attitude dynamics evolve on the non-Euclidean manifold $ SO(3) \times \mathbb{R}^3 $. 
Since these manifolds are locally Euclidean, local stability properties of a closed-loop equilibrium solution can be assessed using standard Lyapunov methods. In addition, the LaSalle invariance result and related Lyapunov results apply to closed-loop vector fields defined on these manifolds. 

However, since the manifolds $ SO(3) $ and $ \mathbb{S}^2 $ are compact, the radial unboundedness assumption cannot be satisfied; consequently, global asymptotic stability cannot follow from a Lyapunov analysis on Euclidean spaces [40], and therefore must be analyzed in alternative ways [19]–[23].\cite[p.43]{Chaturvedi2011}

[40]:
[19]:
[20]:
[21]:
[22]:
[23]:


\cite{Chaturvedi2011} summarizes global results on attitude control and stabilization for a rigid body using continuous time- invariant feedback. The analysis uses methods of geometric mechanics based on the geometry of the special orthogonal group $ SO(3) $ and the two-sphere $ \mathbb{S}^2 $.

%ADD 
%Justify choice of parameters. Up to what level can we push the system? Where can we find more info about domain of attractions.
%How to choose parameters and how to select gains for errors

***************************************\\
%BART 
Bart: Ok, but what are you doing with this? Does this relate to backstepping?\\
Nam: Dit gaat over analyse die je normaal doet mbv Lyapunov om de stabiliteit aan te tonen. 

***************************************\\


***************************************\\
%BART
Bart: Does this include a part about tuning of the controller?\\
Nam: De stabiliteit hangt wel samen met de control parameters. De keuze hiervan is op dit moment arbitrair. De 'juiste' gains kiezen is wellicht zoals eerder besproken overbodig 

***************************************\\

%HOeft waarschijnlijk niet eens een section te zijn, kan kort en bondig
\section*{Summary}

%ADD What is Geometric Control? Differences between other control
In this chapter the control design based on Nonlinear Geometric Control was discussed.

The main difference with other control techniques is that the tracking errors are also defined on manifolds, which allows the design of almost global defined controllers.

Stability analysis is different from a Lyapunov analysis on Euclidean spaces.
%This fact is 

%ADD Why Geometric Control? Why is useful
%PRO
%The proposed control system is robust to switching conditions since each flight mode has almost global stability properties, and it is straightforward to design a complex maneuver of a QR. \cite{Lee2010c}

In order to test the control performance of a load position tracking objective, experiments are defined in the next chapter. 






\chapter{Experiments and Results}\label{ch:results}
%ADD intro: in this chapter
This chapter describes experiments that are done in order to investigate the abilities and performance of a nonlinear Geometric Control design. 
Different cases are defined that describe tracking objectives in order to test the performance of both a simple \a{lqr} control design and a nonlinear Geometric Control design. These results are presented, and this chapter ends with a conclusion based on these results.

\section{Setup}
\paragraph{Model parameters}
The simulations are developed using Matlab and Simulink, using the following system parameters.
\begin{equation}\label{key}
\begin{aligned}
J&=\\
m_Q&=\\
\nonumber
\end{aligned}
\quad
\begin{aligned}
d&=\\
c_{\tau f}&=\\
\nonumber
\end{aligned}
\quad
\begin{aligned}
m_L&=\\
l_L&=\\
\nonumber
\end{aligned}
\end{equation}

\paragraph{LQR Control}
\a{lqr} control is based on small angle assumption. Therefore, a traditional modeling method may represent the rotation matrix with a local coordinate system, for example with a Euler Angle parameterization. An \a{lqr} control design is based on a linearized model of the system, which is derived in Section \ref{app:lqr}, and is shown in Figure \ref{fig:set.lqr}.
\begin{figure}[h!]
	\centering
	\makebox[\textwidth][c]{\includegraphics[width=.45\textwidth]{./StyleStuff/dcsc.png}}
	\caption{LQR control design\label{fig:set.lqr}}
\end{figure}		

%ADD matrix parameters
The tuning parameters of the \a{lqr} controller a chosen as follows
\begin{equation}\label{key}
A=\begin{bmatrix}
content...
\end{bmatrix}
\end{equation}
\begin{equation}\label{key}
B=\begin{bmatrix}
content...
\end{bmatrix}
\end{equation}
\begin{equation}\label{key}
C=\begin{bmatrix}
content...
\end{bmatrix}
\end{equation}

where the state $ \mathbf{x} $ and input $ u $ are defined as 
\begin{align}\label{eq:state}
\textbf{x}&=\begin{bmatrix}
\textbf{q}\\
\mathbf{\dot{q}}
\end{bmatrix}\\
\mathbf{q}&=\begin{bmatrix}
x&y&z&\phi&\theta&\psi&\theta_L&\psi_L
\end{bmatrix}^T\\
\mathbf{\dot{q}}&=\begin{bmatrix}
\dot{x}&\dot{y}&\dot{z}&\dot{\phi}&\dot{\theta}&\dot{\psi}&\dot{\theta}_L&\dot{\psi}_L
\end{bmatrix}^T\\
u&=\begin{bmatrix}
f&M_\phi&M_\theta&M_\psi
\end{bmatrix}^T
\end{align}

\paragraph{Geometric Control}
%ADD chosen parameters GC
The controller gains in Equations \ref{eq:con.M},\ref{eq:con.R},\ref{eq:con.q} are chosen to be
\begin{equation}\label{key}
\begin{aligned}
k_R&=\\
k_\Omega&=\\
\end{aligned}
\quad
\begin{aligned}
k_q&=\\
k_\omega&=\\
\end{aligned}
\quad
\begin{aligned}
k_x&=\\
k_v&=
\nonumber
\end{aligned}
\end{equation}
%ADD chosen model parameters 



%ADD chosen parameters LQR


\subsection{Command Filtering}
%ADD Why is Command Filtering needed? 
Consequence of backstepping control is that inner control loops depend on the the output of outer control loops. The controllers are functions of these generated outputs and their derivatives. This can be calculated analytically, which can be tedious or by estimating with the use of a Command Filter as explained by \cite{Djapic} 
%CHECK Djapic of andere?

A command filter is implemented to compute $ \dot{R}_c, \ddot{R}_c,\dot{q}_c, \ddot{q}_c $, the virtual control command to stabilize the loop within. \cite{Farrell2009}

Examples from \cite{Farrell2008} and \cite{Djapic2008}. 

***************************************\\
%ADD Pro Con Command filter
Easy implementation. Less computational effort.

Less accurate, because filters high frequency signals.

***************************************\\

The load attitude controller generates a commanded QR attitude $ R_c $ and its derivative $ \dot{R}_c $. In the same fashion, the load position controller generates a commanded load attitude $ q_c $ and its derivative $ \dot{q}_c $. The controllers are functions of these commanded signals and their derivatives. Instead of analytic differentiation of these signals, they are obtained by integration by applying a third order low pass filter to the original signals $ R_c^o $ and $ q_c^o $. The transfer function of the original commanded input signal $ X_c^o $ and the filtered output $ X_c $ has the form
\begin{equation}\label{key}
\frac{X_c(s)}{X_c^o(s)}=H(s)=\frac{\omega_{n1}}{s+\omega_{n1}}\cdot\frac{\omega_{n2}^2}{s^2+2\zeta\omega_{n2}s+\omega_{n2}^2}
\end{equation}
Where $ x_c $ is the filtered signal, $ \zeta $ the damping ratio and $ \omega_n $ the undamped natural frequency. See Figure \ref{fig:CF}.
The state space implementation of this third order filter is \cite{Djapic2008}
%CHECK waar dit ook alweer vandaan kwam. Reference in Djapic/Farell -> 3e order voor bacterieen ofzo
\begin{align}\label{key}
\dot{x}_1 &= x_2\\ %dxc
\dot{x}_2 &= x_3\\ %ddxc
\dot{x}_3 &= -(2\zeta \omega_{n2}+\omega_{n1})x_3-(2\zeta\omega_{n1}\omega_{n2}+\omega_{n2}^2)x_2-(\omega_{n1}\omega_{n2}^2)(x_1-x_c^o)
\end{align}
where $ x_1 = x_c$, $ x_2 = \dot{x}_c$ and $ x_3 = \ddot{x}_c$. 

\begin{figure}[h!]
	\centering
	\makebox[\textwidth][c]{\includegraphics[width=.45\textwidth]{./StyleStuff/dcsc.png}}
	\caption{Representation of the command filter\label{fig:CF}}
\end{figure}		

\begin{align}\label{eq:CF}
\frac{x_c}{x_c^o}&=\frac{\omega_{n1}}{s+\omega_{n1}}\cdot\frac{\omega_{n2}^2}{s^2+2\zeta\omega_{n2}s+\omega_{n2}^2}\\
\Rightarrow x_c^{'''}&=-(2\zeta\omega_{n2}+\omega_{n1})x_c^{''}-(2\zeta\omega_{n1}\omega_{n2}+\omega_{n2}^2)x_c^{'}-(\omega_{n1}\omega_{n2} ^2)(x_c-x_c^o)
\end{align}

\section{Experiments}

\a{lqr} is an optimal control strategy and will be used to compare its result to a Nonlinear Geometric Controller.


\subsection{Performance Criteria}
Performance  that can be evaluated for different cases can be specified as the following items.
\begin{outline}
	\1 Step Response
	\2 Settling time (if swing minimization is important)
	\2 Rising time (important if time critical)
	\2 Overshoot (if max swing is critical)
	\2 Steady state error / swing of load (if accuracy is important)
	\2 Max load angle
	
	\1 Disturbance Rejection
	
	\1 Trajectory tracking
	\2 Can we minimize time, while minimizing position error (All Cases)
	\2 Minimum position error (All Cases)
	\2 Maximum amplitude/frequency of wave with respect to stability (Case B)
	
	\1 Computational Effort (?)
\end{outline}

Explain cases, why interesting and what can be expected?\\

\subsection{Case A}
\subsection{Case B}
\subsection{Case C}

\begin{figure}[h!]
	\centering
	\makebox[\textwidth][c]{\includegraphics[width=.2\paperwidth]{./StyleStuff/dcsc.png}}
	\caption{Cases of which the performance could be evaluated \label{fig:routes}}
\end{figure}


\section{Results}
\subsection{Case A}
\subsection{Case B}
\subsection{Case C}


\section{Conclusion}


\chapter{Conclusions and Future Work}\label{ch:conclusion}
\section{Summary}

\section{Thesis Contribution}

\section{Recommendations for Future Work}\label{ch:future}
***************************************\\
Model Validation\\

***************************************\\
\subsection{Modeling Constraints}
There are several techniques to handle input saturation, the most popular ones are anti-windup techniques. Back-calculation is such a method for PID to activate the integrator, is this possible for NL control?

\subsection{Hybrid Modeling}
Switching between several flight modes yields autonomous acrobatic maneuvers. Robust to switching conditions ***why?\\
\begin{figure}[h!]
	\centering
	\includegraphics[width=.45\textwidth]{./StyleStuff/LeeControlscheme.png}
	\caption{\label{fig:}}
\end{figure}		

\subsection{Trajectory Generation}
\subsubsection{Minimum Snap Trajectory Generation}

Trajectory can be generated by solving a \a{QP} via minimum snap generation.

Problem in smaller space with help of differential flatness.

Is able to include constraints.

%\include{future_work}

%LATEX TEST
%\part{TEST PARTS}
%\chapter{Trajectory Generation}\label{ch:trajectory}

***************************************\\
Trajectory is first generated by hand. A simple trajectory that will not push the system to its limits yet\\
Next, trajectory can be generated by solving a \a{QP} via minimum snap generation.

***************************************\\

Trajectory Generation by minimizing Snap Trajectory. QP.\\

\section{Minimum Snap Trajectory Generation}


\section{Conclusion}


%
% First Part
    \part{First Part}

    \chapter{First Real Chapter}

%    This is real chapter for \ac{DUT}, ok? I will explain everything about \gsymb{$\gamma$}{Path Angle}. Next, everything
%    will be explained about the transfer function \lsymb{$H(s)$}{Transfer function}. Also, subscripts and can
    superscripts can be put in the nomenclature \index{nomenclature} list. 
%    \supers{max}{Maximum} 
%    \subs{min}{Minimum} 
    Other things can also
    be added to the nomenclature list of \ac{DUT} 
%    \others{[kts]}{Knots} \others{$^{\circ}$, [deg]}{Degrees}

        \section{First section}

        This is the section. Referring to equations, figures and tables can easily be done by the commands \verb"\eqnref{}",
        \verb"\figref{}" and \verb"\tabref{}".
        \begin{equation}\label{eq:First}
              H(s) = \frac{1}{s+2}
        \end{equation}
        You see? Refer to equations like this \eqnref{eq:First}.
            \subsection{The first subsection}

                \subsubsection[Subsection Short Title]{The first sub-subsection with a very very very long title, but in the table of contents one can only see the short title}

                Nice, ain't it?\index{Nice}

                    \paragraph{A paragraph.}
    \part{Second part}

    \chapter{Second part chapter}
   \begin{figure}
    \caption{this is a very long line to test if the table of
    figures will wrap the line or will continue to go over the
    border  of the page}
    \end{figure}

    \chapter{TEMP second part chapter this is a very long line to test if the table of
    figures will wrap the line or will continue to go over the
    border  of the page}

New chapter gives a full acronym \ac{TU}.

\begin{eqnarray}
% \nonumber to remove numbering (before each equation)
  1 &=& 2\\
  x &=& 5 \\
  y &=& \theta
\end{eqnarray}

    \chapter{Second second part chapter}
    \section{Section}

    This is a test for nomenclature \lsymb{$A(s)$}{Answer function}\\
    \lsymb{$a_f$, $b_f$, $c_f$ and $d_f$}{The variables I am trying to group}\\
    \lsymb{$a_b$}{another variable}\\
\section{Main equations}

\begin{equation}
a=\frac{N}{A}
\end{equation}%

\nomenclature{$a$}{The number of angels per unit area}%
\nomenclature{$N$}{The number of angels per needle point}%
\nomenclature{$A$}{The area of the needle point}%

The equation $\sigma = m a$%
\nomenclature{$\sigma$}{The total mass of angels per unit area}%
\nomenclature{$m$}{The mass of one angel}
follows easily.


	
%============================= Appendices=========================================
\appendix
    \chapter{Appendix}
    
    \section{Derivation of LQR controller}
    
	    \subsection{Modeling}\label{app:model}
%	    \section{Classical Modeling}
	    
	    ***************************************\\
	    A traditional modeling method might represent the rotation matrix with a local coordinate system, for example with a Euler Angle parameterization. 
	    
	    ***************************************\\
	    
	    
	    
	    ***************************************\\
	    A commonly used method for modeling a system is via Newton's law and Lagrangian mechanics. 
	    Based on Euler-Lagrange? $ \rightarrow $ Geometric Mechanics\\
	   
	    ***************************************\\
%	    \subsection{Quadrotor-Load Modeling}	
%	    \subsection{Quadrotor Modeling}
		

	    
	    
	    
	    
%	    ***************************************\\
%	    \begin{equation}
%	    ax = 
%	    \begin{bmatrix}
%	    (f*m*sin(\phi_q)*sin(\psi_q) + f*m_l*sin(\phi_q)*sin(\psi_q) - f*m_l*cos(\theta)^2*sin(\phi_q)*sin(\psi_q) + f*m*cos(\phi_q)*cos(\psi_q)*sin(\theta_q) + f*m_l*cos(\phi_q)*cos(\psi_q)*sin(\theta_q) + l*m*m_l*v_\theta^2*cos(\theta)*sin(\phi) + f*m_l*cos(\phi)^2*cos(\theta)^2*sin(\phi_q)*sin(\psi_q) + l*m*m_l*v_\phi^2*cos(\theta)^3*sin(\phi) - f*m_l*cos(\phi_q)*cos(\psi_q)*cos(\theta)^2*sin(\theta_q) + f*m_l*cos(\phi)^2*cos(\phi_q)*cos(\psi_q)*cos(\theta)^2*sin(\theta_q) + f*m_l*cos(\psi_q)*cos(\theta)*sin(\phi)*sin(\phi_q)*sin(\theta) + f*m_l*cos(\phi)*cos(\phi_q)*cos(\theta_q)*cos(\theta)^2*sin(\phi) - f*m_l*cos(\phi_q)*cos(\theta)*sin(\phi)*sin(\psi_q)*sin(\theta_q)*sin(\theta))/(m*(m + m_l))\\
%	    (l*m*m_l*v_\theta^2*sin(\theta) - f*m_l*cos(\psi_q)*cos(\theta)^2*sin(\phi_q) - f*m*cos(\psi_q)*sin(\phi_q) + f*m*cos(\phi_q)*sin(\psi_q)*sin(\theta_q) + f*m_l*cos(\phi_q)*cos(\theta)^2*sin(\psi_q)*sin(\theta_q) + l*m*m_l*v_\phi^2*cos(\theta)^2*sin(\theta) + f*m_l*cos(\phi)*cos(\phi_q)*cos(\theta_q)*cos(\theta)*sin(\theta) - f*m_l*cos(\theta)*sin(\phi)*sin(\phi_q)*sin(\psi_q)*sin(\theta) - f*m_l*cos(\phi_q)*cos(\psi_q)*cos(\theta)*sin(\phi)*sin(\theta_q)*sin(\theta))/(m*(m + m_l))\\
%	    -(g*m^2 + g*m*m_l - f*m*cos(\phi_q)*cos(\theta_q) - f*m_l*cos(\phi_q)*cos(\theta_q) + l*m*m_l*v_\theta^2*cos(\phi)*cos(\theta) + f*m_l*cos(\phi)^2*cos(\phi_q)*cos(\theta_q)*cos(\theta)^2 + l*m*m_l*v_\phi^2*cos(\phi)*cos(\theta)^3 + f*m_l*cos(\phi)*cos(\psi_q)*cos(\theta)*sin(\phi_q)*sin(\theta) - f*m_l*cos(\phi)*cos(\theta)^2*sin(\phi)*sin(\phi_q)*sin(\psi_q) - f*m_l*cos(\phi)*cos(\phi_q)*cos(\theta)*sin(\psi_q)*sin(\theta_q)*sin(\theta) - f*m_l*cos(\phi)*cos(\phi_q)*cos(\psi_q)*cos(\theta)^2*sin(\phi)*sin(\theta_q))/(m*(m + m_l))\\
%	    (- l*m*cos(\theta)*sin(\theta)*v_\phi^2 + f*cos(\psi_q)*cos(\theta)*sin(\phi_q) - f*cos(\phi)*cos(\phi_q)*cos(\theta_q)*sin(\theta) - f*cos(\phi_q)*cos(\theta)*sin(\psi_q)*sin(\theta_q) + f*sin(\phi)*sin(\phi_q)*sin(\psi_q)*sin(\theta) + f*cos(\phi_q)*cos(\psi_q)*sin(\phi)*sin(\theta_q)*sin(\theta))/(l*m)\\
%	    -(f*cos(\phi_q)*cos(\theta_q)*sin(\phi) + f*cos(\phi)*sin(\phi_q)*sin(\psi_q) - 2*l*m*v_\phi*v_\theta*sin(\theta) + f*cos(\phi)*cos(\phi_q)*cos(\psi_q)*sin(\theta_q))/(l*m*cos(theta))
%	    \end{bmatrix}
%	    \end{equation}	    
%	    
%	    ***************************************\\	
	    
	    \subsection{Controller}
	    
    
    \begin{align}\label{eq:state}
    \textbf{x}&=\begin{bmatrix}
    \textbf{q}\\
    \mathbf{\dot{q}}
    \end{bmatrix}\\
    \mathbf{q}&=\begin{bmatrix}
    x&y&z&\phi&\theta&\psi&\theta_L&\psi_L
    \end{bmatrix}^T\\
    \mathbf{\dot{q}}&=\begin{bmatrix}
    \dot{x}&\dot{y}&\dot{z}&\dot{\phi}&\dot{\theta}&\dot{\psi}&\dot{\theta}_L&\dot{\psi}_L
    \end{bmatrix}^T\\
    u&=\begin{bmatrix}
    f&\tau_\phi&\tau_\theta&\tau_\psi
    \end{bmatrix}^T
    \end{align}
    
    \begin{align}\label{eq:ss}
    \mathbf{\dot{x} }&=A\mathbf{x}+Bu\\
    y&=C\mathbf{x}+Du
    \end{align}
    
    %CHECK if needed?
    \begin{equation}\label{key}
    \mathbf{\dot{x}}=\textbf{f}(\mathbf{x,u})
    \end{equation}
    
    %ADD reference
    From Equation follows
    \begin{align}\label{key}
    %CHECK this equations
    \ddot{x}&=-\frac{f}{m}\\
    \ddot{y}&=\\
    \ddot{z}&=
    \end{align}
    
    The mathematical model is linearized around the following operating points
    \begin{align}\label{key}
    \bar{\mathbf{x}}&=\begin{bmatrix}\bar{x}&\bar{y}&\bar{z}&\textbf{0}_{1\times13}\end{bmatrix}^T\\
    \bar{\textbf{u}}&=\begin{bmatrix}
    (m_Q+m_L)g&0&0&0
    \end{bmatrix}^T
    \end{align}
    
    %CHECK are these asssumptions correct?
    Assuming small angles, the following holds
    \begin{align}\label{key}
    \text{for } \gamma &= \phi, \theta, \psi, \theta_L, \psi_L\\
    sin(\gamma)&\simeq \gamma\\
    cos(\gamma)&\simeq 1\\
    \dot{\gamma} &\simeq 0\\
    F &\simeq (m_Q+m_L)g
    \end{align}
    
    \begin{equation}\label{key}
    A=\frac{\partial \textbf{f}(x,u)}{\partial x}\mid _{	x=\bar{x},u=\bar{u}	}
    \end{equation}
    \begin{equation}\label{key}
    B=\frac{\partial \textbf{f}(x,u)}{\partial u}\mid _{	x=\bar{x},u=\bar{u}	}
    \end{equation}
    
    \begin{equation}\label{key}
    u=-K\left[\textbf{x}_{des}(t)-\textbf{x}(t)\right] 
    \end{equation}
    
    \section{Derivation of Equations of motion}
		\subsection{Load Dynamics}\label{sec.app:loaddyn}
		%PROOF prop.3 Sreenath2013a. Also Sreenath2013b?
		%DEFINE e_3 / R / f 
		
		Let \lsymb{$ x_{CM} $}{Position \lsymb{$ CM $}{Center of Mass} of \a{qr}-Load system} denote the position of the center of mass of the combined Quadrotor-Load system, expressed in \IF. Which can be found by
		\begin{align}\label{eq:CM}
		\begin{split}
		m_Q(x_Q-x_{CM})+m_L(x_L-x_{CM})&=0\\
		(m_Q+m_L)x_{CM}&=m_Qx_Q+m_Lx_L
		\end{split}
		\end{align}
		Applying the laws of motion to (\ref{eq:CM}) and inserting (\ref{eq:xQ2xL}) gives the 
		\begin{align}\label{key}
		\begin{split}
		(m_Q+m_L)\ddot{x}_{CM}&=fRe_3 - (m_Q+m_L)ge_3\\
		%&=fRe_3 - (m_Q+m_L)ge_3\\
		%(m_Q+m_L)\ddot{x}_{CM}&=m_Q\ddot{x}_Q + m_L\ddot{x}_L\\
		%\\
		%m_Q\ddot{x}_Q+m_L\ddot{x}_L&=fRe_3 - (m_Q+m_L)ge_3\\
		%m_Q(\ddot{x}_L-L\ddot{q})+m_L\ddot{x}_L&=fRe_3 - (m_Q+m_L)ge_3\\
		(m_Q+m_L)(\ddot{x}_L+ge_3)&= fRe_3+m_QL\ddot{q}
		\end{split}
		\end{align}
		
		%ADD derivation of ddq (TANG2014)

    
    \section{An appendix section}
\begin{equation}\label{key}
Test equation
\end{equation}
    \subsection{A \matlab Listing}

    \lstset{language=matlab}
    \lstinputlisting{test.m}
    
%    \subsection{An appendix subsection with C++ Listing}
%
%    \lstset{language=C++}
%    \lstinputlisting{test.c}    

%    \chapter{Yet another appendix}
%
%    \section{Test section (again?)}
%
%    Ok, all is well.

%============================= Back matter =========================================
\backmatter
	% Bibliography
	\bibliographystyle{ieeetr}
	\bibliography{library}
	
	%CHECK all entries for uniformity 	
	
	\printbib{library}
	
	% Index
	\printindex

	% Nomenclature
	\printnomenclature
\printacronyms

\begin{acronym}[\hspace{0.8in}] % 0.8in is also used by the nomenclature
	\acro{3mE}[3\textlarger{m}E]{Mechanical, Maritime and Materials Engineering}%
	\acro{DCSC}{Delft Center for Systems and Control}%
	\acro{TU}[TU D\textlarger{elft}]{Delft University of Technology}%

	\acro{cmp}[CMP]{Cooperative Manipulation Problem}
	\acro{atp}[ATP]{Aerial Towing Problem}
	\acro{QR}[QR]{Quadrotor}
	\acro{uav}[UAV]{Unmanned Aerial Vehicles}
	
	\acro{lti}[LTI]{Linear Time Invariant}
	\acro{ltvmpc}[LTV MPC]{Linear Time Variant MPC}
	\acro{mpc}[MPC]{Model Predictive Control}
	\acro{mipc}[MIPC]{Mixed-Integer Predictive Control}	
	\acro{nmpc}[NMPC]{Nonlinear Model Predictive Control}
	\acro{mld-mpc}[MLD-MPC]{Mixed Logical Dynamical - Model Predictive Control}
	\acro{lq}[LQ]{Linear Quadratic}
	\acro{lqr}[LQR]{Linear Quadratic Regulator}

	\acro{mld}[MLD]{Mixed Logical Dynamical}
	\acro{pwa}[PWA]{Piecewise Affine}

	\acro{aladin}[ALADIN]{Alternating Direction Inexact Newton method}
	\acro{hysdel}[HYSDEL]{HYbrid Systems DEscription Language}	
	\acro{dha}[DHA]{Discrete Hybrid Automata}
	
	\acro{lp}[LP]{Linear Programming}
	\acro{dp}[DP]{Dynamic Programming}	
	\acro{nlp}[NLP]{Nonlinear Program}
	\acro{qp}[QP]{Quadratic Programming}
	\acro{qcp}[QCP]{Quadratic Constrained Programming}	
	\acro{qcqp}[QCQP]{Quadratically Constrained Quadratic Program}
	
	\acro{sqp}[SQP]{Sequential Quadratic Programming}
	\acro{ip}[IP]{Interior Point}
	
	\acro{mip}[MIP]{Mixed Integer Programming}
	\acro{milp}[MILP]{Mixed Integer Linear Programming}
	\acro{miqp}[MIQP]{Mixed Integer Quadratic Programming}
	\acro{miqcp}[MIQCP]{Mixed-integer Quadratically Constrained Programming}

	
	\acro{cog}[CoG]{Center of Gravity}
	
	\acro{ocp}[OCP]{Optimal Control Problem}
	\acro{dae}[DAE]{Differential-Algebraic Equation}
	
	\acro{pd}[PD]{Proportional-Derivative}
	\acro{pid}[PID]{Proportional-Integral-Derivative}
	
	\acro{rls}[RLS]{Recursive Least Squares}



\end{acronym}
	% Acronyms
%    \nonumchap{Acronyms}    
%    \begin{acronym}[XXXXX]% Note: replace XXXXX by the longest acronym
%                          %       in your list.
%        \acro{DUT}{Delft University of Technology}%
%    \end{acronym}%

\end{document}
