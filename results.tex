\chapter{Results}\label{ch:results}
\section{Case A}
In Figure \ref{fig:set.caseAres} the results are shown for the Nonlinear Geometric Controller. The desired and actual load trajectory, and the position error are shown in Figure \ref{fig:AxL} and Figure \ref{fig:AexL}, respectively.
From this can be seen that a small steady state error remains in the z-direction. However, $ (e_x,e_v)=(0,0) $ is exponentially attractive. 
%CHECK kan deze error nog verholpen worden? Gain ex ev omhoog? 
Figure \ref{fig:AeR} and Figure \ref{fig:Aeq} also show that the tracking error for both attitude and angular velocity are exponentially attractive.
Figure \ref{fig:APsiR} confirms Equation \ref{eq:con.PsiRconv} and likewise, Figure \ref{fig:APsiq} confirms Equation \ref{eq:con.Psiqconv} and it can be seen that both the configuration error $ \Psi_R $ on $ SO(3) $ and $ \Psi_q$ on $ \mathbb{S}^2 $ is smaller than 2 and converges to zero along the trajectory.
\begin{figure}[h!]
	\centering
	\makebox[.49\textwidth][c]{\subfloat[][]{\includegraphics[width=.5\textwidth]{\dir{LPOSQRL-xL40}}\label{fig:AxL}}}	
	\makebox[.49\textwidth][c]{\subfloat[][]{\includegraphics[width=.5\textwidth]{\dir{LPOSQRL-exL40}}\label{fig:AexL}}}	
	\makebox[.49\textwidth][c]{\subfloat[][]{\includegraphics[width=.5\textwidth]{\dir{LPOSQRL-eR40}}\label{fig:AeR}}}
	\makebox[.49\textwidth][c]{\subfloat[][]{\includegraphics[width=.5\textwidth]{\dir{LPOSQRL-eq40}}\label{fig:Aeq}}}
	\makebox[.49\textwidth][c]{\subfloat[][]{\includegraphics[width=.5\textwidth]{\dir{LPOSQRL-PsiR40}}\label{fig:APsiR}}}
	\makebox[.49\textwidth][c]{\subfloat[][]{\includegraphics[width=.5\textwidth]{\dir{LPOSQRL-Psiq40}}\label{fig:APsiq}}}
	\caption{Results Nonlinear Geometric Control Case A \label{fig:set.caseAres}}
\end{figure}	

\section{Case B}
\ref{fig:set.caseBres}
%CHECK right pictures?
\begin{figure}[h!]
	\centering
	\makebox[.49\textwidth][c]{\subfloat[][]{\includegraphics[width=.45\textwidth]{\dir{LPOSQRL-xL41}}\label{fig:}}}	
\makebox[.49\textwidth][c]{\subfloat[][]{\includegraphics[width=.45\textwidth]{\dir{LPOSQRL-exL41}}\label{fig:}}}	
\makebox[.49\textwidth][c]{\subfloat[][]{\includegraphics[width=.5\textwidth]{\dir{LPOSQRL-eq41}}\label{fig:}}}
\makebox[.49\textwidth][c]{\subfloat[][]{\includegraphics[width=.5\textwidth]{\dir{LPOSQRL-eR41}}\label{fig:}}}
\makebox[.49\textwidth][c]{\subfloat[][]{\includegraphics[width=.45\textwidth]{\dir{LPOSQRL-PsiR41}}\label{fig:}}}
\makebox[.49\textwidth][c]{\subfloat[][]{\includegraphics[width=.45\textwidth]{\dir{LPOSQRL-Psiq41}}\label{fig:}}}
	\caption{Results Nonlinear Geometric Control Case B \label{fig:set.caseBres}}
\end{figure}		

\section{Case C}
Despite the fact that the \a{qr} is moving side to side, the upward force can still be controlled sufficiently to track the desired height. Figure \ref{fig:CxL} shows the desired load position, and Figure \ref{fig:CexL} shows that the error is mainly the overshoot in the x-direction, due to the fast desired swinging motion. 
\ref{fig:set.caseCres}
%CHECK right pictures?
\begin{figure}[h!]
	\centering
	\makebox[.49\textwidth][c]{\subfloat[][]{\includegraphics[width=.45\textwidth]{\dir{LPOSQRL-xL41}}\label{fig:CxL}}}	
\makebox[.49\textwidth][c]{\subfloat[][]{\includegraphics[width=.45\textwidth]{\dir{LPOSQRL-exL41}}\label{fig:CexL}}}	
\makebox[.49\textwidth][c]{\subfloat[][]{\includegraphics[width=.5\textwidth]{\dir{LPOSQRL-eq41}}\label{fig:Ceq}}}
\makebox[.49\textwidth][c]{\subfloat[][]{\includegraphics[width=.5\textwidth]{\dir{LPOSQRL-eR41}}\label{fig:CeR}}}
\makebox[.49\textwidth][c]{\subfloat[][]{\includegraphics[width=.45\textwidth]{\dir{LPOSQRL-PsiR41}}\label{fig:CPsiR}}}
\makebox[.49\textwidth][c]{\subfloat[][]{\includegraphics[width=.45\textwidth]{\dir{LPOSQRL-Psiq41}}\label{fig:CPsiq}}}
	\caption{Results Nonlinear Geometric Control Case C \label{fig:set.caseCres}}
\end{figure}	

\section{Conclusion}\label{set:set.con}
%CHECK 
%What can we learn and conclude from different performance comparisons

%What is its value of nonlinear control compared to linear control

%No restrictions on rotor direction. is it possible to turn 2 ways?

%Could it be interesting for a real-time on-board controller
%Considering the computational power of an on-board processor is limited
%computational effort vs what?

The nonlinear geometric controller depends on feed forward terms that are obtained from the desired trajectories. 
Trajectory generation approaches exist that are able to generate the required desired position, velocity and acceleration by 
however it is possible to compute these with trajectory generating algorithms too.

The controllers are functions of the computed tracking references $ q_c, R_c $ and their derivatives. These terms are approximated by a command filter, which means that the accuracy decreases because high frequency terms are filtered.