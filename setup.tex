\chapter{Experiment}\label{ch:setup}
The experimental procedure is explained in Section \ref{sec:set.proc}. 
It is discussed what experiments can be done in order to investigate the potential of nonlinear geometric control.
In addition, a comparison will be made between the performances of the Nonlinear Geometric Controller and a linear \a{lqr} controller.

The controllers are tested on their ability to track a desired load trajectory. 
Section \ref{sec:set.traj} presents trajectories that create situations with different challenges.
It is discussed what could be expected from these experiments, and the differences in performance of the controllers is discussed.
%in what way a comparison can be made between the performance of the controllers.

In Section \ref{sec:set.setup} the experimental setup is discussed. 
The model parameters for the \a{qr}-Load system are presented, as well as the controller parameters for both nonlinear Geometric controller and \a{lqr} controller.
The notion of a backstepping command filter is made to explain a mathematical simplification in the experiments.
% are presented and for comparison of 
%The experiments are done with two 
%model- and controller parameters are presented.
%The trajectory tracking experiments are explained and discussed in Section \ref{sec:set.traj}. 
%The results of the experiments are presented and discussed, and final conclusions are made in Chapter \ref{ch.results}.

%ADD 1 wat brengt mij tot keuze trajectories 
%ADD 2 trajectories
%ADD 3 waar ga ik de resultaten op checken
%\paragraph{Performance}

%\begin{outline}
%	\1 Step Response
%	\2 Settling time (if swing minimization is important)
%	\2 Rise time (important if time critical)
%	\2 Overshoot (if max swing is critical)
%	\2 Steady state error / swing of load (if accuracy is important)
%%	\1 Max load angle
%%	\1 Disturbance Rejection
%	\1 Trajectory tracking
%	\2 How do the errors evolve along the trajectory
%	\2 What is the maximum error during the trajectory
%%	\2 Can we minimize time, while minimizing position error (All Cases)
%%	\2 Minimum position error (All Cases)
%%	\2 Maximum amplitude/frequency of wave with respect to stability (Case B)
%%	\1 Computational Effort (?)
%\end{outline}
\section{Procedure}\label{sec:set.proc}
Performance of both Nonlinear Geometric Control and \a{lqr} control can be evaluated by comparing their ability to track a load trajectory with minimal error. 
In linear control however, a linearized model is obtained by assuming small angles of both load and \a{qr} around an equilibrium point. 
The model is obtained by assuming the system in equilibrium when the \a{qr} is in hover position with the load hanging directly underneath it.
As a result, the linearized model does not allow direct reference tracking of the load position. 

The \a{lqr} cost function allows control of the inputs $ f $ and $ M $, and the states which define the \a{qr} position, \a{qr} attitude and load attitude. 
Therefore, load control is only possible in means of minimizing the load swing. This fact illustrates an important difference between the use of a linear and a nonlinear model. 

The experiments describe a desired load trajectory $ x_{L,d}(t) $, which is required to be smooth for Geometric Control, such that feed forward terms can be generated and implemented. In this thesis the desired load paths are generated by hand, and the required velocity and acceleration is calculated by a command filter, of which the details are described in Section \ref{sec:set.setup}.

The experiments done with the \a{lqr} controller will apply reference tracking of the \a{qr} position, which is  based on the desired load trajectories that are used for the nonlinear Geometric controller. When assuming small angles and minimal load swing, the \a{qr} position is approximately a cable length above the predefined desired load position. 
Note that this will not allow a direct comparison of the load trajectory tracking, nevertheless this will illustrate the main differences 
between the controllers with the same purpose of load transportation, but with a different approach.
%are defined in a different way.
%with different ob
%in means of
%and allow conclusions to be made about the 
%in load attitude
%exposed to trajectories 
%where large angles are required to track fast maneuvers


% Since it is not possible for \a{lqr} to apply reference tracking for the load trajectory, 
%\a{lqr} is a linear optimal control strategy and will be used to compare its result to a Nonlinear Geometric Controller.





\section{Trajectories}\label{sec:set.traj}
%\begin{description}\label{key}
%\item[Step response] How does the system respond on a smooth step in load trajectory?
%\item[Settling time] How long does it take to stabilize after 
%\end{description}

What observations can be made in order to adapt the controller properties that improve performance of the test cases.
%ADD
Description of tests that apply on all cases.

\subsection{Case A}
%ADD 
%Explain cases, why interesting and what can be expected?\\
In this case a smooth step-like trajectory is generated to transport the load from a starting position along the direction of the x-asis to the final position. In a regular step function the system is subjected to a sudden input. 
The stability of the system can be investigated by observing whether it is able to reach a stationary final state, and how fast this can be reached. 
It can be seen whether the system responds with an overshoot and how fast the response is, when the controller tries to track the trajectory. 

%whether the final steady state can be reached after tracking the trajectory, and how fast this can be reached.
Figure \ref{fig:set.caseA} shows the desired trajectory over time, and a three dimensional representation.
\begin{figure}[h!]
	\centering
	\makebox[.49\textwidth][c]{\subfloat[][ \label{fig:AxLd}]{\includegraphics[width=.45\textwidth]{\dir{LPOSQRL-xLdes40}}}}
	\makebox[.49\textwidth][c]{\subfloat[][ \label{fig:AxLdplot}]{\includegraphics[width=.45\textwidth]{\dir{LPOSQRL-xLdesplot40}}}}
	\caption{Desired Load Position Case A\label{fig:set.caseA}}
\end{figure}		

%ADD inputs f M
%ADD QR attitude

\subsection{Case B}
PLANNING: make case to test limits on \a{qr} angles while tracking load trajectory. Is nonlinear GC useful for such aggressive maneuvers?

%Figure \ref{fig:set.caseB} shows the desired trajectory over time, and a three dimensional representation.
%\begin{figure}[h!]
%	\centering
%	\makebox[.49\textwidth][c]{\subfloat[][ \label{fig:}]{\includegraphics[width=.45\textwidth]{\dir{LPOSQRL-xLdes41}}}}
%	\makebox[.49\textwidth][c]{\subfloat[][ \label{fig:}]{\includegraphics[width=.45\textwidth]{\dir{LPOSQRL-xLdesplot41}}}}
%	\caption{Desired Load Position Case B\label{fig:set.caseB}}
%\end{figure}		

\subsection{Case C}
For this case a trajectory is generated to test multiple disciplines. 
The trajectory has the shape of a sine wave that moves along the y-axis and varies in amplitude in the direction of the x-axis, while going up and down in the direction of the z-axis.
%Changing velocities are required to track 
The changing amplitude of the trajectory that moves from side to side, requires varying velocities to 'keep up' with the trajectory. 
It can be expected that the nonlinear geometric control allows large \a{qr} angles, whereas the \a{lqr} will possible fail to deviate far from the equilibrium point. 

Figure \ref{fig:set.caseC} shows the desired trajectory over time, and a three dimensional representation.
\begin{figure}[h!]
	\centering
	\makebox[.49\textwidth][c]{\subfloat[][ \label{fig:}]{\includegraphics[width=.45\textwidth]{\dir{LPOSQRL-xLdes41}}}}
	\makebox[.49\textwidth][c]{\subfloat[][ \label{fig:}]{\includegraphics[width=.45\textwidth]{\dir{LPOSQRL-xLdesplot41}}}}
	\caption{Desired Load Position Case C\label{fig:set.caseC}}
\end{figure}		


\section{Setup}\label{sec:set.setup}
\paragraph{Model parameters}
%ADD chosen model parameters 
The simulations are developed using Matlab and Simulink, using the system parameters found in Table \ref{tab:set.par}.

\begin{table}[h!]
	\centering
	\begin{tabular}{|l|ll|l|}
		\hline
		\textbf{Parameter}&\textbf{Value}&&\textbf{Description}\\
		\hline
		$ m_Q $&4.34& $ kg $&Quadrotor Mass\\
		$ m_L $&0.1 &$ kg $&Load Mass\\
		$ l $&0.315& $ m $&Arm length from \a{qr} \a{cm} to rotor\\
		$ L $&0.7 &$ m $& Cable Length\\
		$ I_{xx} $&0.0820&$kgm^2 $&Quadrotor Inertia about x-axis\\
		$ I_{yy} $&0.0845&$kgm^2 $&Quadrotor Inertia about y-axis\\
		$ I_{zz} $&0.1377&$kgm^2 $&Quadrotor Inertia about z-axis\\
%		$ d $&&&Drag Constant\\
%		$ b $&&&Thrust Constant\\
%		$ c_{\tau_f} $&&& Constant	\\
		\hline	
	\end{tabular}
	\caption{Modeling Parameters}
	\label{tab:set.par}
\end{table}

\paragraph{LQR Control}
%ADD why LQR, what is good for / known for
% tuning / easy tuning / 
\acf{lqr} control uses an algorithm to obtain a state-feedback controller, minimizing a cost function depending on the states and weight factors. 
An \a{lqr} design is shown in Figure \ref{fig:set.lqr}
\begin{figure}[h!]
	\centering
	\makebox[\textwidth][c]{\includegraphics[width=.45\textwidth]{./StyleStuff/dcsc.png}}
	\caption{LQR control design\label{fig:set.lqr}}
\end{figure}

\a{lqr} control is based on a small angle assumption. Therefore, a traditional modeling method may represent the rotation matrix with a local coordinate system, for example with an Euler Angle parameterization. 
A continuous time linearized model of the system used in this controller is represented in the following form 
\begin{align}\label{eq:ss}
\mathbf{\dot{x} }&=A\mathbf{x}+Bu\\
y&=C\mathbf{x}+Du
\end{align}
where $ \mathbf{x} $ is the state vector and $ u $ is the input vector, defined as follows
\begin{align}\label{eq:state}
%	\textbf{x}&=\begin{bmatrix}
%		\textbf{q}\\
%		\mathbf{\dot{q}}
%	\end{bmatrix}\\
%	\mathbf{q}&=\begin{bmatrix}
%		x&y&z&\phi&\theta&\psi&\phi_L&\theta_L
%	\end{bmatrix}^T\\
%	\mathbf{\dot{q}}&=\begin{bmatrix}
%		\dot{x}&\dot{y}&\dot{z}&\dot{\phi}&\dot{\theta}&\dot{\psi}&\dot{\phi}_L&\dot{\theta}_L
%	\end{bmatrix}^T\\
\mathbf{x}&=\begin{bmatrix}
x&y&z&\phi&\theta&\psi&\phi_L&\theta_L&\dot{x}&\dot{y}&\dot{z}&\dot{\phi}&\dot{\theta}&\dot{\psi}&\dot{\phi}_L&\dot{\theta}_L
\end{bmatrix}^T\\
	u&=\begin{bmatrix}
		f&M_\phi&M_\theta&M_\psi
	\end{bmatrix}^T
\end{align}
%ADD figure that define phiL and thetaL
where $ \phi_L $ and $ \theta_L $ are the angle of rotation of the load about the x-axis and y-axis in \BF, respectively.
The derivation of $ A, B, C, D $ can be found in Section \ref{app:lqr}. 

Using \texttt{Matlab} command \texttt{lqr(A,B,Q,R)}, an optimal gain matrix $ K $ is calculated, such that the state-feedback law $ u=-K\mathbf{x} $ minimizes the quadratic cost function defined as
\begin{equation}\label{key}
J(u)=\int_{0}^{\infty}(\mathbf{x}^TQ\mathbf{x}+u^TRu)dt
\end{equation}
%where $ Q $ and $ R $ denote weight matrices that penalize the states and inputs in the cost function. 
The weight matrices $ Q $ and $ R $ that define the effects of the states and inputs in the cost function, and the calculated gain matrix $K $ can be found in Section \ref{app:lqr}. 

%CHECK dubbelop?
%where the state $ \mathbf{x} $ and input $ u $ are defined as 
%\begin{align}\label{eq:state}
%\textbf{x}&=\begin{bmatrix}
%\textbf{q}\\
%\mathbf{\dot{q}}
%\end{bmatrix}\\
%\mathbf{q}&=\begin{bmatrix}
%x&y&z&\phi&\theta&\psi&\theta_L&\psi_L
%\end{bmatrix}^T\\
%\mathbf{\dot{q}}&=\begin{bmatrix}
%\dot{x}&\dot{y}&\dot{z}&\dot{\phi}&\dot{\theta}&\dot{\psi}&\dot{\theta}_L&\dot{\psi}_L
%\end{bmatrix}^T\\
%u&=\begin{bmatrix}
%f&M_\phi&M_\theta&M_\psi
%\end{bmatrix}^T
%\end{align}

\paragraph{Geometric Control}
%ADD chosen parameters GC
The chosen controller gains in Equations \ref{eq:con.M},\ref{eq:con.R},\ref{eq:con.q} can be found in Table \ref{tab:set.gains}.

\begin{table}[h!]
	\centering
	\begin{tabular}{|l|l|}
		\hline
		\textbf{Gain}&\textbf{Value}\\
		\hline
		$ k_R $&\\
		$ k_\Omega $&\\
		$ k_q $&\\
		$ k_\omega $&\\
		$ k_x $&\\
		$ k_v $&\\	
		\hline
	\end{tabular}
	\caption{Controller Gains}
	\label{tab:set.gains}
\end{table}

\paragraph{Command Filtering}
%ADD Pro Con Command filter
%Easy implementation. Less computational effort.
%Less accurate, because filters high frequency signals.
%CHECK
%Examples from \cite{Farrell2008} and \cite{Djapic2008}. 
A consequence of a backstepping control approach, is that it also increases the order of the states. The inner control loops become a function of the commanded signals and their higher derivatives, which are generated by an outer loop.
In the earlier presented control design, the load attitude controller generates a commanded QR attitude $ R_c $ and its derivative $ \dot{R}_c $. In the same fashion, the load position controller generates a commanded load attitude $ q_c $ and its derivative $ \dot{q}_c $. 
Instead of analytic differentiation of these terms, which can be tedious and require high computational costs, these values can be obtained with the use of a Command Filter, which is explained in more detail in \cite{Farrell2008}. 

The basic idea is that the command signal is pre-filtered by a low pass filter and generates an estimation of the derivatives of the commanded signal. 
In this thesis a backstepping command filter of third order is applied to compute $ \dot{R}_c, \ddot{R}_c,\dot{q}_c, \ddot{q}_c $. 
The transfer function of the original commanded input signal $ X_c^o $ and the filtered output $ X_c $ has the form
\begin{equation}\label{key}
\frac{X_c(s)}{X_c^o(s)}=H(s)=\frac{\omega_{n1}}{s+\omega_{n1}}\cdot\frac{\omega_{n2}^2}{s^2+2\zeta\omega_{n2}s+\omega_{n2}^2}
\end{equation}
Where $ \zeta $ is the damping ratio and $ \omega_n $ the undamped natural frequency. See Figure \ref{fig:set.CF} and \ref{fig:app.CF}.
The filter has the following state space representation
%CHECK waar dit ook alweer vandaan kwam. Reference in Djapic/Farell -> 3e order voor bacterieen ofzo
\begin{align}\label{key}
\dot{x}_1 &= x_2\\ %dxc
\dot{x}_2 &= x_3\\ %ddxc
\dot{x}_3 &= -(2\zeta \omega_{n2}+\omega_{n1})x_3-(2\zeta\omega_{n1}\omega_{n2}+\omega_{n2}^2)x_2-(\omega_{n1}\omega_{n2}^2)(x_1-x_c^o)
\end{align}
where $ x_1 = x_c$, $ x_2 = \dot{x}_c$ and $ x_3 = \ddot{x}_c$. 
%CHECK  nodig?
%\begin{figure}[h!]
%	\centering
%	\makebox[\textwidth][c]{\includegraphics[width=.45\textwidth]{./StyleStuff/cf.png}}
%	\caption{Representation of the command filter\label{fig:set.CF}}
%\end{figure}		

%The controllers are functions of these commanded signals and their derivatives. Instead of analytic differentiation of these signals, they are obtained by integration by applying a third order low pass filter to the original signals $ R_c^o $ and $ q_c^o $. 
%The state space implementation of this third order filter is \cite{Djapic2008}
%\begin{align}\label{eq:CF}
%\frac{x_c}{x_c^o}&=\frac{\omega_{n1}}{s+\omega_{n1}}\cdot\frac{\omega_{n2}^2}{s^2+2\zeta\omega_{n2}s+\omega_{n2}^2}\\
%\Rightarrow x_c^{'''}&=-(2\zeta\omega_{n2}+\omega_{n1})x_c^{''}-(2\zeta\omega_{n1}\omega_{n2}+\omega_{n2}^2)x_c^{'}-(\omega_{n1}\omega_{n2} ^2)(x_c-x_c^o)
%\end{align}





